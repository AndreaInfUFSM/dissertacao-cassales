%\listfiles
\documentclass[tg]{mdtufsm}
% um tipo específico de monografia pode ser informado como parâmetro opcional:
%\documentclass[tese]{mdtufsm}
% a opção `openright' pode ser usada para forçar inícios de capítulos
% em páginas ímpares
% \documentclass[openright]{mdtufsm}
% para gerar uma versão frente-e-verso, use a opção 'twoside':
% \documentclass[twoside]{mdtufsm}

\usepackage[T1]{fontenc}        % pacote para conj. de caracteres correto
\usepackage{fix-cm} %para funcionar corretamente o tamanho das fontes da capa
\usepackage{times, color, xcolor}       % pacote para usar fonte Adobe Times e cores
\usepackage[utf8]{inputenc}   % pacote para acentuação
\usepackage{graphicx}  % pacote para importar figuras
\usepackage{amsmath,latexsym,amssymb} %Pacotes matemáticos
\usepackage[%hidelinks%, 
            bookmarksopen=true,linktoc=none,colorlinks=true,
            linkcolor=black,citecolor=black,filecolor=magenta,urlcolor=blue,
            pdftitle={Desenvolvimento de um escalonador sensível ao contexto para o Apache Hadoop},
            pdfauthor={Guilherme Weigert Cassales},
            pdfsubject={Trabalho de Graduação},
            pdfkeywords={Apache Hadoop, escalonador, sensível ao contexto, Informática, UFSM}
            ]{hyperref} %hidelinks disponível no pacote hyperref a partir da versão 2011-02-05  6.82a
%Nesse caso, hidelinks retira os retângulos em volta dos links das referências

%Margens conforme MDT 7ª edição, arrumar diretamente no mdtufsm.cls para funcionar a opção twoside *PENDENTE*
\usepackage[inner=30mm,outer=20mm,top=30mm,bottom=20mm]{geometry} 
\usepackage{epstopdf}
\usepackage{graphicx}


%==============================================================================
% Se o pacote hyperref foi carregado a linha abaixo corrige um bug na hora
% de montar o sumário da lista de figuras e tabelas
% Se o pacote não foi carregado, comentar a linha %
%==============================================================================
\input{macros/bugcaption}

%==============================================================================
% Identificação do trabalho
%==============================================================================
\title{Desenvolvimento de um escalonador sensível ao contexto para o Apache Hadoop}

\author{Cassales}{Guilherme Weigert}

\course{Curso de Ciência da Computação}
\altcourse{Curso de Ciência da Computação}

\institute{Centro de Tecnologia}
\degree{Bacharel em Ciência da Computação}

% Número do TG (verificar na secretaria do curso)
% Para mestrado deixar sem opção dentro do {}
\trabalhoNumero{}

%Orientador
\advisor[Profª.]{Drª.}{Charão}{Andrea Schwertner}
%Se for uma ``orientadora'' descomentar a linha baixo
\orientadoratrue

%Avaliadores (Banca)
\committee[Prof. Dr.]{Stein}{Benhur de Oliveira}{UFSM}
\committee[Profª. Drª.]{Barcelos}{Patrícia Pitthan de Araújo}{UFSM}


% a data deve ser a da defesa; se nao especificada, são gerados
% mes e ano correntes
\date{13}{Novembro}{2013}

%Palavras chave
\keyword{Apache Hadoop} 
\keyword{Escalonador}
\keyword{Sensibilidade ao Contexto}

%%=============================================================================
%% Início do documento
%%=============================================================================
\begin{document}

%%=============================================================================
%% Capa e folha de rosto
%%=============================================================================
\maketitle

%%=============================================================================
%% Catalogação (obrigatório para mestrado) e Folha de aprovação
%%=============================================================================
%Somente obrigatório para dissertação, para TG, remover as linhas	77	%
%Como a CIP vai ser impressa atrás da página de rosto, as margens inner e outer	
%devem ser invertidas.
%\newgeometry{inner=20mm,outer=30mm,top=30mm,bottom=20mm}	
%\makeCIP{nomedoautor@gmail.com} %email do autor		
%\restoregeometry

%Se for usar a catalogação gerada pelo gerador do site da biblioteca comentar as linhas
%acima e utilizar o comando abaixo
%\includeCIP{CIP.pdf}

%folha de aprovação
\makeapprove


%%=============================================================================
%% Resumo
%%=============================================================================
\begin{abstract}

Hoje em dia, o volume de dados gerados é muito maior do que a capacidade de processamento dos computadores. Como solução para esse problema, algumas tarefas podem ser paralelizadas ou distribuidas. O \emph{framework Apache Hadoop} \cite{Hadoop}, é uma delas e poupa o programador as terefas de gerenciamento, como tolerância à falhas, particionamento dos dados entre outros.
Um problema no escalonador do \emph{Apache Hadoop} é que seu foco é em ambientes homogêneos, o que muitas vezes não é possível de se manter. O foco deste trabalho é implementar um novo escalonador que seja sensível ao contexto e que leve em conta as capacidades físicas de cada máquina na hora de distribuir as tarefas submetidas.

Nowadays the volume of data generated by the services provided for end users, is way larger than the processing capacity of one computer alone. As a solution to this problem, some tasks can be parallelized. The Apache Hadoop framework \cite{Hadoop}, is one of these parallelized solutions and it spares the programmer of management tasks such as fault tolerance, data partitioning, among others.
One problem on this framework is the scheduler, which is designed for homogeneous environments. It is worth to remember that maintaining a homogeneous environment is somewhat difficult today, given the fast development of new, cheaper and more powerful hardware. This work focuses on altering the Capacity Scheduler, in order to make it more context-aware towards resources on the cluster since, the original scheduler assumes that every node on cluster has the same amount of memory and therefore may set the resources too high or too low. 
This work managed to insert code on NodeManager methods so that it will collect data on every node and send it to CapacityScheduler, which will adjust the allocation values according to the total cluster resources.






\end{abstract}


%% Lista de Ilustrações (opc)
%% Lista de Símbolos (opc)
%% Lista de Anexos e Apêndices (opc)

%%=============================================================================
%% Lista de figuras (comentar se não houver)
%%=============================================================================
%%\listoffigures

%%=============================================================================
%% Lista de tabelas (comentar se não houver)
%%=============================================================================
%%\listoftables

%%=============================================================================
%% Lista de Apêndices (comentar se não houver)
%%=============================================================================
%\listofappendix

%%=============================================================================
%% Lista de Anexos (comentar se não houver)
%%=============================================================================
%\listofannex

%%=============================================================================
%% Lista de abreviaturas e siglas
%%=============================================================================
 %o parametro deve ser a abreviatura mais longa
%\begin{listofabbrv}{UbiComp}
%   \item [BNF] \textit{Backus-Naur Form}
%   \item [UbiComp] Computação Ubíqua
%\end{listofabbrv}


%%=============================================================================
%% Lista de simbolos (opcional)
%%=============================================================================
%Simbolos devem aparecer conforme a ordem em que aparecem no texto
% o parametro deve ser o símbolo mais longo
%\begin{listofsymbols}{teste}
%  \item [$\varnothing$] vazio
%  \item [$\Gamma$]  Gama
%  \item [$\forall$] Para todo
%\end{listofsymbols}

%%=============================================================================
%% Sumário
%%=============================================================================
\tableofcontents


%%=============================================================================
%% Início da dissertação
%%=============================================================================
\setlength{\baselineskip}{1.5\baselineskip}

%Adiciona cada capitulo
\chapter{Introduction}
\label{chap:Introduction}
One of the major IT companies nowadays, known as Google \cite{Google}, had the initial idea of a way to process a huge data volume generated by its servers. This approach would later be known as MapReduce, built by two separate steps, Map and Reduce, each step based on a functional language function. At the same time, a Yahoo! \cite{Yahoo} led project was starting the implementation of MapReduce for its own system, which would then become a whole new project, named Apache Hadoop \cite{Hadoop}.

Today the Apache Hadoop framework has a very active community of both developers and users, however there are some characteristics that weren't changed from the day the framework was first designed. Among these characteristics there is one very detrimental and prone to bad performance issues, which is the focus on homogeneous environments. It is known that maintaining a totally homogeneous environment is harder and harder as the time passes, requiring either a huge initial investment or a huge effort in order to replace faulty hardware without changing the component capability.

The MapReduce's task performance inside Hadoop is tightly tied to the scheduler \cite{CASH}. Since it is an open source project, it is possible to change the scheduler aiming to make it capable of better adapting to heterogeneity while at the same time presenting a performance improvement.

A key characteristic in Hadoop's transition to heterogeneous environments is the context-aware capability. The definition of context can vary from one application to another, but as a rule of thumb it is some information that the application can use as base for decision making. When an application is context-aware, it will detect and adapt to the changes in the environment \cite{Manuele}.

In the present work, the context to which the application will have to adapt is related to the physical configuration of the machines that compose the Hadoop cluster, allowing the scheduler to work with real data collected from the machines and not suppositions as the default Hadoop configuration implies. In a more complex degree, the present work is just a part of a bigger project called PER-MARE \cite{PER-MARE}, which has the objective of adapting to more environment variations, as the insertion and removal of nodes in real time.


%-------------------------------------------------------------------
\section{Objective}

The main objective of this work is to improve Hadoop through a context-aware scheduling, which will provide better performance and adaptation on heterogeneous environments.

%-------------------------------------------------------------------
\section{Motivation}

Today, some processing tasks that used to be made through huge mainframes and servers are gradually transitioning to big clusters, which are composed of computers with more accessible prices and easily bought in the market. 

Even though the Apache Hadoop framework has clusters as its target, it was designed and implemented under a specific assumption. The framework's better performance is achieved when it is running on a homogeneous cluster, in other words, when all nodes have the same resources. The problem is that given today's hardware development, it might take the system to a point where it is not possible or at least not profitable to maintain cluster homogeneity. 

Since the default configuration of the scheduler tells that every node on the cluster has the same amount of resources, if a more powerful node is inserted all that extra capacity will be wasted. This happens because the cluster will not collect the real configuration, but the parameter set in a XML file as the node capacity. The opposite is also troublesome, if a less powerful node is inserted, the cluster will use it as if it had more potential, possibly overloading that node with more tasks that it can handle and causing errors or performance issues.

The present work is relevant, as it's objective is based on adaptation and improvement of an already existent technology. With the improved scheduler, not only will the Apache Hadoop clusters have a possibility to improve cluster's resource utilization, as the framework itself will be better prepared and capable of adapting to new heterogeneous environments in an easier and smoother way.
\chapter{Fundamentos e Revisão de Literatura}

%-------------------------------------------------------------------
Este capítulo destina-se à definição de conceitos teóricos sobre as ferramentas e paradigmas utilizados no trabalho, os quais são listados a seguir: \emph{Framework Apache Hadoop}, \emph{MapReduce}, bem como trabalhos relacionados.

%-------------------------------------------------------------------
\section{Hadoop}
A origem do \emph{framework Apache Hadoop}, vem de outro projeto da \emph{Apache} \cite{Apache}, o \emph{Apache Nutch} \cite{Nutch}, que era um motor de buscas na \emph{web} com código livre iniciado em 2002. Porém o projeto encontrava problemas devido a sua arquitetura. Em 2003 quando a \emph{Google} publicou um artigo descrevendo a arquitetura utilizado no seu sistema de arquivos distribuídos, chamado GFS, os desenvolvedores viram que uma arquitetura semelhante resolveria o problema de escalabilidade do \emph{Nutch}.

Em 2004 os desenvolvedores do \emph{Nutch} começaram a implementar a ideia e o resultado foi nomeado \emph{Nutch Distributed Filesystem} (NDFS). A medida que o projeto avançava ele foi tomando proporções cada vez maiores, até que em 2006 foi criado um novo projeto pois os avanços já ultrapassavam o propósito do \emph{Nutch}, o novo projeto foi nomeado \emph{Hadoop}. O \emph{framework Hadoop} tem o propósito de facilitar o processamento distribuído através do paradigma do \emph{MapReduce}.

\subsection{Arquitetura geral do \emph{Apache Hadoop}}
De maneira geral é possível separar o \emph{Apache Hadoop} em duas partes, as quais são denominadas \emph{HDFS (Hadoop Distributed File System)} e \emph{YARN (Yet Another Resource Negotiator)}. A Figura \ref{fig:ArqGeral} fornece uma visão de como o \emph{framework} é estruturado.

\begin{figure}[hbtn]
   \centering
   \includegraphics[width=9cm]{figuras/Figura08-HadoooArchGeral.png}
   \caption{Arquitetura geral do \emph{Apache Hadoop}}
   \label{fig:ArqGeral}
\end{figure}

O HDFS é a parte responsável pelo armazenamento dos dados necessários para que os jobs sejam executados, em outras palavras, é um grande HD distribuído como indica sua denominação (\emph{Distributed File System}). O HDFS é o componente que irá fazer a replicação de tolerância a falhas, distribuição dos dados de acordo com o que cada nó irá processar, entre outras atribuições.

A outra metade do \emph{Apache Hadoop}, o YARN, é responsável pelo processamento dos \emph{jobs} submetidos ao \emph{cluster}. É dentro do YARN que as tarefas de \emph{MapReduce} são executadas, consequentemente o YARN é o componente que gerencia todos os recursos do \emph{cluster}. 

\subsubsection{HDFS}
O HDFS é em grande parte responsável pelo bom desempenho do \emph{Apache Hadoop}, pois é encarregado com a tarefa de não sobrecarregar a rede com transferência de arquivos. No HDFS o acesso a arquivo é sempre local, isso quer dizer que cada nó receberá a parte do arquivo referente a sua carga de trabalho, evitando assim replicação desnecessária além da básica para segurança e tolerância a falhas.
Um problema dessa abordagem é que o \emph{Hadoop} possui uma latência muito alta, sendo desaconselhável o uso do \emph{Hadoop} em aplicações críticas ou de tempo real. O HDFS pode ser subdivido em dois serviços, \emph{NameNode} e \emph{DataNode}, responsáveis pelo gerenciamento dos dados a nível de \emph{cluster} e gerenciamento dos dados a nível local, respectivamente. A Figura \ref{fig:ArqHDFS} apresenta um esquema básico da arquitetura do HDFS.

\begin{figure}[hbtn]
   \centering
   \includegraphics[width=8cm]{figuras/Figura07-HDFS.png}
   \caption{Arquitetura geral do HDFS \cite{HDFS}}
   \label{fig:ArqHDFS}
\end{figure}

\subsubsection{YARN}
O YARN é a parte do \emph{Apache Hadoop} responsável pela execução do \emph{MapReduce}, portanto à ele cabem as tarefas de gerenciamento e execução do processamento. Ao tornar a tarefa de processamento totalmente independente das tarefas de armazenamento, o \emph{Apache Hadoop} abre muitas possibilidades para sua utilização. Assim como o HDFS, o YARN pode ser subdividido em 2 serviços, \emph{ResourceManager} e \emph{NodeManager}, responsáveis pelo gerenciamento dos recursos no sistema e pelo gerenciamento dos recursos locais, respectivamente. A Figura \ref{fig:ArqYARN} apresenta um esquema básico da arquitetura do YARN.

\begin{figure}[hbtn]
   \centering
   \includegraphics[width=12cm]{figuras/Figura06-YarnArch.png}
   \caption{Arquitetura geral do YARN \cite{YARN}}
   \label{fig:ArqYARN}
\end{figure}

Embora não demonstrado na imagem, cada serviço possui diversos módulos internos, por exemplo o \emph{ResourceManager} possui o Escalonador e o \emph{ApplicationsManager}, os quais ainda podem ser divididos em sub-módulos menores. O presente trabalho busca apresentar uma nova solução para a maneira de escalonamento empregada no \emph{Hadoop} que se adapte melhor ao ambiente.

\subsection{Configuração do ambiente de execução do \emph{Hadoop}}
Um ambiente corretamente configurado do \emph{Hadoop} possui alguns pré-requisitos além dos nós acessíveis entre si em uma rede. Cada nó deve ter em sua instalação do \emph{Hadoop} vários arquivos xml, que são responsáveis pela configuração das MVs do \emph{Hadoop} naquela máquina.
Para conhecimento, esses arquivos são: \emph{core-site.xml, yarn-site.xml, mapred-site.xml} e \emph{hdfs-site.xml}. Cada um desses arquivos terá a configuração de um serviço do \emph{Hadoop}. O arquivo \emph{hdfs-site.xml} por exemplo é responsável pela configuração do HDFS naquela máquina.
É importante salientar que esta configuração em nenhum momento é realizada automaticamente pelo \emph{Hadoop} e que o usuário deve configurar cada nó separadamente.

%%incluir apendices/anexos com as configurações utilizadas?

\subsection{\emph{MapReduce}}
O paradigma de \emph{MapReduce}, já citado várias vezes no presente trabalho, divide o processamento em duas etapas. Essas duas etapas são derivadas das funções \emph{Map} e \emph{Reduce} das linguagens funcionais, e assim como nas funções oiginais, elas tem funcionamento baseado em tuplas de chave e valor. Um ciclo de aplicação típico é a função \emph{Map} receber um arquivo de entrada e buscar os valores procurados pela aplicação, então formar tuplas de chave e valor. Feitas as tuplas, a função \emph{Map} manda o resultado para a \emph{Reduce} onde as chaves serão processadas e reduzidas a dados mais significativos. A grande vantagem do \emph{Hadoop} é que dado um ambiente corretamente configurado, o programador pode focar sua atenção à resolução das tarefas pelo paradigma do \emph{MapReduce} e não em como o trabalho será distribuído.

\section{Sensibilidade ao contexto}
Dada a interligação dos sistemas hoje em dia, já é possível notar alguma sensibilidade ao contexto na maioria deles. Ao acessar um site por um dispositivo móvel, o site automaticamente irá carregar sua versão \emph{mobile}, a qual foi projetada para estes dispositivos, ou quando os navegadores utilizam dados de localidade para oferecer produtos, entre outros exemplos de utilização do contexto.

Segundo \cite{Zakaria}, sensibilidade ao contexto na computação se refere a habilidade de uma aplicação de detectar e responder as mudanças no ambiente de execução. O que leva a seguinte definição feita por \cite{Baldauf}, onde ele afirma que um sistema sensível ao contexto é capaz de adaptar suas operações ao contexto atual sem intervenção explicita do usuário e portanto aumentar sua usabilidade e eficácia.

Partindo dessas duas afirmações vem a dúvida sobre o que seria o contexto, portanto a definição do contexto é fundamental para que haja um entendimento da sensibilidade ao contexto \cite{Manuele}. O contexto pode assumir diversos significados dependendo da situação que se encontra, \cite{Dey} define contexto como qualquer informação que pode ser utilizada para caracterizar a situação de uma entidade (pessoa, lugar ou objeto) considerado relevante para a interação entre usuário e aplicação.

Geralmente informações de contexto são utilizadas para a melhoria de performance de um sistema ou algoritmo, portanto estima-se ser possível melhorar a execução do \emph{Apache Hadoop} através da utilização dessa técnica. 

Embora existam diversas maneiras de se utilizar essas informações de contexto para melhorar a performance do \emph{MapReduce}, \cite{Manuele} cita três exemplos de como isso pode ser feito, os quais se resumem em: configuração automática dos nós durante a instalação, gerenciamento de entrada e saída de nós do \emph{cluster} e finalmente na distribuição de tarefas feita pelo escalonador de acordo com a disponibilidade de recursos e tarefas já em execução. A terceira maneira apresentada é a maneira que corresponde à abordagem utilizada nesse trabalho.

\section{Escalonadores para Hadoop}
Uma dos principais componentes do \emph{Hadoop} é o escalonador, componente responsável pela distribuição do trabalho no ambiente. Além dos escalonadores disponibilizados juntamente com o próprio \emph{Hadoop}, existem outras implementações que buscam solucionar uma necessidade específica que os escalonadores padrões não oferecem suporte.

\subsection{\emph{Hadoop Internal Scheduler}}
O escalonador padrao do \emph{Hadoop} foi implementado visando suportar apenas a submissão de tarefas em lote. Nesse escalonador, a primeira tarefa recebida é a primeira executada, formando uma fila para as subsequentes. Apesar de simples este escalonador também suporta cinco níveis de prioridade, porém a escolha da próxima tarefa nunca deixará o tempo de submissão completamente de fora.

\subsection{\emph{Fair Scheduler}}
Utilizado para computar tarefas pequenas em lote que possuam os mesmos dados de entrada, utilizando um escalonamento em dois níveis para distribuir recursos igualitáriamente \cite{FairScheduler}. O nível superior, geralmente aloca filas para cada usuário, utilizando um algoritmo justo com pesos. O segundo nível aloca os recursos dentro de cada fila, e utiliza um algoritmo igual ao \emph{Internal Scheduler}.

\subsection{\emph{Capacity Scheduler}}
Este escalonador surgiu para os casos onde um ambiente \emph{Hadoop} é dividido entre várias empresas ou possui partes distribuídas em diversos locais sob responsabilidade de mais de um dono. Ele é focado em garantias de que uma quantidade mínima de recursos será disponibilizada a qualquer momento que um de seus usuários decidir utilizar o \emph{Hadoop}. O benefício decorre que organizações diferentes possuem picos de processamento em horas diferentes, portanto as organizações que estão utilizando o \emph{Hadoop} irão se aproveitar da capacidade ociosa das outras.

\section{Trabalhos relacionados}

Foi feita uma pesquisa bibliográfica com objetivo de analisar os trabalhos que já haviam sido desenvolvidos envolvendo o \emph{Hadoop} e que se propunham a alterar ou adaptar o escalonador. Além disso, buscou-se identificar quais técnicas eram as mais utilizadas e em cima de quais objetivos o trabalho foi desenvolvido. A seguir encontram-se os trabalhos relacionados e um breve resumo sobre a proposta, contexto utilizado e objetivo esperado com as alterações.

\begin{itemize}
\item CASH (\emph{Context Aware Scheduler for Hadoop}) \cite{CASH}, nesse trabalho o objetivo dos autores é de melhorar o rendimento geral do \emph{cluster}. Eles partem da hipótese de que grande parte dos jobs são periódicos e executados no mesmo horário, além de possuírem características de uso de CPU, rede, disco etc. semelhantes. O trabalho ainda leva em consideração que com o passar do tempo os nós tendem a ficar mais heterogêneos. Com a intenção de solucionar esses problemas e baseados nessas hipóteses, foi implementado um escalonador que classifica tanto os jobs como as máquinas com relação ao seu potencial de CPU e E/S, podendo então distribuir os jobs para máquinas que tem uma configuração apropriada para sua natureza.

\item LATE (\emph{Longest Approximation Time to End}) \cite{LATE}, seguindo o que o nome sugere, nesse tabalho a informação de contexto é referente ao tempo estimado de término da \emph{task} baseado numa heurística que faz a relação de tempo decorrido e \emph{score}. Essa informação é usada também para gerar um limiar de quando uma \emph{task} é lenta o suficiente para indicar sitomas de erros e então iniciar uma nova em outra máquina possívelmente mais rápida. O objetivo do trabalho era de reduzir o tempo de resposta em \emph{clusters} grandes que executam muitos \emph{jobs} de pequena duração.

\item \emph{A Dynamic MapReduce Scheduler for Heterogeneous Workloads} \cite{DMRSHW}, aqui os autores também utilizam a técnica de classificar os \emph{jobs} e máquinas de acordo com a quantidade de E/S ou CPU. E assim como no CASH, o principal objetivo é a melhora de rendimento no \emph{cluster}. Uma das diferenças, no entanto, é que essa implementação utiliza um escalonador com três filas.

\item SAMR (\emph{A Self-adaptative MapReduce}) \cite{SAMR}, essa implementação segue a mesma ideia do LATE, onde a informação de contexto é referente ao cálculo do progresso de uma \emph{task} para identificar se é necessário lançar outra task igual ou não. Porém essa solução varia o cálculo do progresso de acordo com informações do ambiente em que a \emph{task} está sendo executada, seu principal objetivo é a redução do tempo de execução das \emph{tasks}. Para que sejam utilizadas informações do ambiente, o algoritmo leva em consideração informações históricas contidas em cada nó, e ajusta o peso de cada estágio do processamento.

\item COSHH (\emph{A Classification and Optimization based Scheduler for Heterogeneous Hadoop Systems}) \cite{COSHH}, um pouco mais abrangente das demais soluções apresentadas, essa solução leva em consideração informações não especificadas sobre o sistema. Seu ganho de performance se dá a partir da classificação dos \emph{jobs} em classes, ele então faz uma busca por máquinas que se encaixem nessa mesma classe. Essa busca é feita por um algoritmo que reduz o tamanho do espaço de busca para melhorar o rendimento. O objetivo dessa solução é a melhora do tempo médio em que os \emph{jobs} são completados, além de oferecer uma boa performance quando utilizando somente o fatia mínima, além de proporcionar uma distribuição justa.

\item \emph{Quincy} \cite{Quincy}, diferentemente de todos os outros trabalhos, essa solução não foi desenvolvida visando somente o \emph{Hadoop} mas ainda assim é aplicável ao mesmo. Possuindo como objetivo melhorar o desempenho geral de um \emph{cluster}, utiliza como informação de contexto a distribuição de recursos e modifica a maneira tradicional de tratamento desses. Ao utilizar um modo mais dinâmico do que o convencional, a solução mapeia os recursos num grafo de capacidades e demandas e calcula o escalonamento ótimo a partir de uma função global de custo.

\item \emph{Improving MapReduce Performance through Data Placement in heterogeneous Hadoop Clusters} \cite{IMRPDPHHC}, buscando melhorar a performance de \emph{jobs} que possuam muito processamento de dados através da melhor distribuição desses dados, essa solução utiliza principalmente a localidade dos dados como informação para tomada de decisões. O ganho de performance é dado pelo rebalanceamento dos dados nos nós, deixando nós mais rápidos com mais dados. Isso diminui o custo de \emph{jobs} especulativos e de transferência de dados pela rede.
\end{itemize}

Após estudo dos trabalhos, nota-se que muitos deles tem por objetivo a diminuição do tempo de resposta ou a melhoria do rendimento de maneira geral, os quais diferem dos objetivos do presente trabalho, que na verdade busca proporcionar uma melhor adaptação do \emph{Hadoop} a um ambiente heterogêneo. 

Constata-se também que há uma diversidade de contextos levados em consideração, contudo é possível identificar temas recorrentes como : a classificação dos \emph{jobs} e dos nós quanto ao potencial de E/S ou de CPU, a avaliação do progresso da \emph{task} na decisão de lançar ou não uma nova \emph{task} especulativa.
\chapter{Métodos e Desenvolvimento}
\label{cap:desen}
Este capítulo descreve as etapas de desenvolvimento e as metodologias empregadas neste trabalho. Buscaram-se estratégias sem intrusão ou grandes modificações nas políticas de escalonamento já implementadas pelo \textit{framework}. Nas Seções \ref{sec:collector} e \ref{sec:zookeeper} são apresentados em maior detalhe os coletores de contexto e a ferramenta de comunicação distribuída utilizados neste trabalho, respectivamente. Finalmente, na Seção \ref{sec:solucao} faz-se uma análise em profundidade da solução implementada comparando-a com o comportamento \textit{default} do Hadoop em ambientes compartilhados.

%-------------------------------------------------------------------
\section{Coletores de Contexto}
\label{sec:collector}
Após um estudo aprofundado dos escalonadores do Hadoop, ficou claro que o Capacity Scheduler já está estruturado de maneira a oferecer escalabilidade e um desempenho satisfatório. Porém, este escalonador possui um ponto fraco, o qual é exposto quando utilizado em um ambiente compartilhado. O Capacity Scheduler só recebe informações sobre os Node Managers no momento da inicialização destes, e ainda, a informação é obtida de um arquivo de configuração. A importância de que as informações sobre os Node Managers estejam atualizadas decorre da dependência intrínseca do escalonamento com a disponibilidade de recursos, na qual uma informação errada pode influenciar o algoritmo de maneira prejudicial e diminuir o desempenho das tarefas. Com base nestas observações, o primeiro passo para a solução é a coleta de dados sobre os recursos dos Node Managers, ou seja, a adição de sensibilidade ao contexto.

\subsection{Sensibilidade ao Contexto}
\label{sec:ctx}
Contexto é definido por \cite{Dey} como qualquer informação que pode ser utilizada para caracterizar a situação de uma entidade (pessoa, lugar ou objeto) considerada relevante para a interação entre usuário e aplicação. Um exemplo de utilização de contexto é quando um usuário acessa um site por meio de um dispositivo móvel e o site carrega automaticamente a versão \textit{mobile}, a qual possui alterações que aumentam a compatibilidade com o tipo de dispositivo sendo utilizado. Outra situação semelhante ocorre quando dados de localidade são utilizados para melhorar os resultados de um motor de buscas, mostrando primeiro os resultados mais relacionados com a região ou idioma do usuário. Nota-se que nas duas situações a aplicação utiliza dados específicos, o tipo do dispositivo e a localização do usuário, coletados no momento da execução e utiliza-os para adaptar o seu funcionamento de maneira a oferecer maior conforto ou usabilidade ao usuário. 

Sendo assim, se uma aplicação é capaz de coletar informações sobre a situação do sistema no qual está sendo executada, ela é capaz de coletar informações de contexto. Porém a simples coleta não aumenta, de fato, o desempenho desta aplicação, é necessário que a aplicação seja capaz de responder às mudanças detectadas no ambiente. Esta capacidade de detecção e reação é caracterizada como sensibilidade ao contexto por \cite{Maamar} e vai ao encontro da definição de \cite{Baldauf} em que o sistema deve detectar mudanças e adaptar suas operações sem intervenção explícita do usuário, aumentando assim a usabilidade e eficácia da aplicação.

\subsection{Implementação}
Em um primeiro momento, buscou-se diminuir a dependência nos arquivos XML para a configuração dos recursos dos Node Managers com intuito de facilitar a configuração inicial dos nós e futuramente utilizar este mesmo mecanismo para a inclusão do suporte ao compartilhamento dos recursos. Para isto, fez-se necessário o desenvolvimento de um conjunto de coletores de contexto capazes de coletar de maneira eficiente os recursos do nó em questão no momento da inicialização do serviço Node Manager e, consequentemente, passar a informação correta sobre os recursos no momento do registro no Resource Manager.

Embora esta adição já facilite a utilização do Hadoop em um \textit{cluster} de natureza heterogênea, ainda não é suficiente para que o Hadoop seja utilizado eficientemente em um \textit{cluster} compartilhado. A razão para esta afirmação é que, embora a informação esteja correta no momento de inicialização, é possível que, durante a execução das aplicações de \textit{MapReduce}, os recursos comecem a ser utilizados por outros usuários, diminuindo a capacidade disponível para o Hadoop. O problema causado pela utilização de alguns nós por outros usuários sem a devida atualização dos dados no escalonador é a sobrecarga destes nós, uma vez que o Hadoop irá tentar utilizar memória e processadores que já estão alocados à outros processos.
% causando, nos piores casos, \textit{Swap} da memória.
%TODO Se deixar isso, colocar dados de swap nos testes

Com objetivo de solucionar o problema causado pelo compartilhamento, é necessário que a coleta de dados ocorra não apenas na inicialização do Node Manager mas ao longo da execução do serviço em intervalos periódicos. Para isso optou-se pela utilização de uma \textit{thread} para a coleta e transmissão, se necessária, dos dados.

O coletor escolhido para a tarefa foi o coletor desenvolvido pelo projeto PER-MARE \cite{Collector}, o qual utiliza a interface padrão do Java para monitoramento, OperatingSystemMXBean \cite{MXBean}. A implementação deste coletor de contexto é baseada em uma interface, uma classe abstrata e as classes de coleta dos recursos desejados. Devido ao seu projeto, coletores de novas informações podem ser facilmente criados, aumentando assim a quantidade de informação disponível para o escalonador.

A interface OperatingSystemMXBean, possibilita o acesso às informações do sistema no qual a JVM está sendo executada. Uma vez que a classe abstrata implementa esta interface, todas suas herdeiras poderão utilizá-la.

As classes utilizadas neste trabalho fazem a coleta de memória física disponível e processadores disponíveis, os recursos suportados por padrão no Hadoop. É possível visualizar o digrama de classes na Figura \ref{fig:collectorUML}, onde estão presentes alguns exemplos de possíveis coletores a serem utilizados.

%TODO fazer outro DC
\begin{figure}[!hbtn]
   \centering
   \includegraphics[width=15cm]{figuras/CollectorUML2.pdf}
   \caption{Diagrama de classes dos coletores de contexto}
   \label{fig:collectorUML}
\end{figure}

Cada instância de Node Manager possui um conjunto de coletores (um para memória e um para processadores), os quais realizam a coleta num intervalo pré-definido. Os coletores são executados por uma \textit{thread} independente e possuem um intervalo de 30 segundos entre as coletas para não causar sobrecarga ou interrupção no  processamento das tarefas \textit{Map} e \textit{Reduce}.

%-------------------------------------------------------------------
\section{Comunicação Distribuída}
\label{sec:zookeeper}

Para que a informação coletada pelos coletores de contexto possa afetar o escalonamento é necessário que exista uma maneira para os Node Managers transmitirem os dados atualizados ao escalonador, a ferramenta escolhida para esta tarefa foi o ZooKeeper.

\subsection{ZooKeeper}
O ZooKeeper é um projeto da Apache e fornece ferramentas eficientes, confiáveis e tolerantes à falha para a coordenação de sistemas distribuídos \cite{Hunt2010}. Inicialmente, o ZooKeeper foi implementado como um componente do Hadoop e virou um projeto próprio conforme cresciam suas funcionalidades e sua utilização em outras aplicações. 

A arquitetura utilizada no ZooKeeper é a de cliente-servidor, sendo o servidor o próprio ZooKeeper (chamado de \textit{ensamble}), enquanto a aplicação que o está utilizando assume o papel de cliente. Os dados do ZooKeeper ficam armazenados em \textit{zNodes}, abstrações que podem ser tanto um \textit{container} de dados quanto de outros \textit{zNodes}, e formam um sistema de arquivos hierárquico que pode ser comparado à estrutura de uma árvore. Para garantir a consistência deste sistema de arquivos o ZooKeeper utiliza operações de escrita linearizáveis, as quais são obrigatoriamente processadas pelo servidor líder que é, então, encarregado de propagar as mudanças para os demais participantes do \textit{ensamble} \cite{Pham}.

Um dos recursos do ZooKeeper que oferece grande utilidade é o \textit{Watcher}, uma interface que permite aos clientes o monitoramento de certos \textit{zNodes}. Quando um \textit{Watcher} é registrado como monitor de um \textit{zNode}, ele pode ser configurado para monitorar a alteração dos dados do \textit{zNode}, a criação/remoção de \textit{zNodes} filhos, ou ainda para qualquer tipo de alteração no \textit{zNode} e seus filhos. Quando um \textit{zNode} sofre uma alteração que o \textit{Watcher} está monitorando uma \textit{callback} é disparada para que o cliente faça o processamento desejado. O disparo desta \textit{callback} é um evento único, forçando o programador a re-inserir um \textit{Watcher} no \textit{zNode} para continuar monitorando-o \cite{HadoopBook}.

No contexto deste trabalho, os serviços do ZooKeeper são utilizados para monitorar as informações de contexto coletadas nos nós escravos e transmiti-las para o escalonador. Sendo que a comunicação é feita através de processos que atualizam e monitoram o conteúdo de \textit{zNodes}.

%----------------------
\subsection{Implementação}
A flexibilidade oferecida através dos \textit{zNodes} permite que qualquer estrutura de dado seja inserida como informação, porém existem alguns detalhes importantes no gerenciamento de \textit{Watchers} que podem impactar na eficácia da solução. Por esta razão optou-se por utilizar um \textit{zNode} para cada Node Manager, o que permite realizar um controle mais rígido das atualizações e possibilita uma maneira fácil e rápida tanto para a inserção das informações por parte dos Node Managers quanto para o monitoramento dos dados pelo escalonador.

Na solução adotada cada \textit{zNode} contém informação de apenas um Node Manager e para cada \textit{zNode} existe um Watcher, sendo que cada \textit{Watcher} é uma thread pertencente ao escalonador. Qualquer alteração em um \textit{zNode} dispara uma \textit{callback} para seu \textit{Watcher} específico, o qual lê os novos dados do \textit{zNode} e atualiza a informação no escalonador sobre os recursos daquele nó. A utilização do \textit{Watcher} permite que o escalonador realize operações apenas quando necessário e não desperdice tempo percorrendo os \textit{zNodes} quando não houverem atualizações a serem feitas. A estrutura adotada no trabalho podes ser visualizada na Figura \ref{fig:zk}.


%Qualquer alteração na tabela dispara uma \textit{callback} no escalonador, que por sua vez percorre a tabela à procura das modificações e realiza as alterações pertinentes nas informações sobre os recursos. Esta abordagem permite que o escalonador realize operações apenas quando necessário e não desperdice tempo acessando a tabela quando não houverem atualizações a fazer. A estrutura adotada no trabalho podes ser visualizada na Figura \ref{fig:zk}.


\begin{figure}[!hbtn]
   \centering
   \includegraphics[width=15cm]{figuras/Zookeeper.pdf}
   \caption{Estrutura do ZooKeeper}
   \label{fig:zk}
\end{figure}

A solução escolhida gera dois papéis para os nós, o papel de monitoramento e o papel de atualização, os quais são explicados em detalhe a seguir.

O papel de \textbf{Monitoramento} é desempenhado pelo escalonador, o qual possui uma \textit{thread} \textit{Watcher} que monitora um \textit{zNode}. Este \textit{zNode} inicial servirá como \textit{container} para os demais \textit{zNodes}, os quais armazenarão as informações de contexto referentes aos nós escravos do \textit{cluster}. O \textit{Watcher} do \textit{zNode} inicial permite ao escalonador ser notificado quando novos \textit{zNodes} forem inseridos na estrutura. Quando a \textit{callback} do \textit{Watcher} é acionada, a \textit{thread} recebe todos os \textit{zNodes} filhos daquele que está sendo monitorado e após identificar o novo \textit{zNode}, inicia uma nova \textit{thread Watcher} para monitorar o novo \textit{zNode} identificado. Desta forma, cada \textit{zNode} será monitorado por uma \textit{thread Watcher} única. Um possível problema da técnica utilizada é discutido, juntamente com a solução adotada para solucioná-lo, na descrição do papel de Atualização.

Uma alternativa à utilização de um \textit{zNode} para cada nó escravo, seria a utilização de uma estrutura que permitisse a inserção de todos os dados dos nós escravos, como uma Tabela Hash. Embora, em um primeiro momento, o maior impecilho para a utilização desta técnica possa parecer o limite padrão do tamanho da informação que um \textit{zNode} comporta (1 Mb), o problema é, na verdade, relacionado com a utilização de uma única \textit{thread Watcher}. 

Como todos os nós escravos são, geralmente, inicializados ao mesmo tempo, as coletas de dado são realizadas em períodos semelhantes, estes períodos serão denominados neste estudo como fase de coleta. Durante estas fases, é possível que muitas atualizações sejam feitas em um pequeno espaço de tempo, gerando um risco à consistência das informações, uma vez que o ZooKeeper não fornece garantias de que todas as alterações notificarão a \textit{thread Watcher}. Este comportamento ocorre devido à possibilidade de notificações serem disparadas antes da re-inserção do \textit{Watcher} no \textit{zNode} que contém a Tabela Hash. Esta falha só ocorre quando a segunda notificação é enviada antes que a primeira termine de ser processada, e existem duas situações que isto pode ocorrer. A primeira situação não apresenta grandes riscos à consistência das informações, pois a notificação perdida  ocorre entre duas notificações normais na mesma fase de coleta. Sendo assim, a notificação seguinte fará com que o escalonador ajuste os valores de recursos dos dois nós. Já a segunda situação apresenta riscos à consistência das informações, pois ocorre quando não há uma notificação normal na mesma fase de coleta após a notificação perdida, ou seja, a notificação perdida só terá suas informações atualizadas no escalonador na próxima fase de coleta, a qual ocorrerá em aproximadamente 30 segundos. Este período que o escalonador opera com informações inconsistentes pode causar o lançamento de \textit{containers} que irão exceder o limite dos nós caso um novo usuário tenha iniciado a utilização do nó, gerando sobrecarga e lentidão tanto na aplicação do novo usuário como na aplicação MapReduce.

%	o papel de monitor é desempenhado pelo escalonador. O escalonador implementa a interface Watcher, fornecida pelo ZooKeeper, e monitora o \textit{zNode} que contém a Tabela Hash. No momento da inicialização o escalonador cria um \textit{zNode} e insere uma Tabela Hash vazia como informação, então finalmente, inicia o monitoramento do \textit{zNode}. Quando a \textit{callback} é acionada, a \textit{thread} percorre a tabela em busca de qual informação foi atualizada, atualiza os dados do escalonador e reinicia o monitoramento.
	
O papel de \textbf{Atualização} é realizado pelos Node Managers, os quais lançam, no momento de sua inicialização, uma \textit{thread} responsável por fazer a coleta de dados e, caso houver alteração com relação à ultima coleta, atualizar os dados no \textit{zNode}. A coleta dos dados é realizada a cada 30 segundos, intervalo que corresponde à média de tempo observada em \textit{containers} executados em um \textit{cluster} de funcionamento normal. Com objetivo de evitar o envio de dados por alterações naturais do sistema e que não impactariam no escalonamento foi implementada uma política de atualização, na qual atualizações só serão realizadas quando as variações alterarem o limite de \textit{containers} que podem ser inicializados no nó, seja pela carga de CPU ou disponibilidade de memória. Esta política permite ao escalonador otimizar seu poder de adaptabilidade sem desperdiçar tempo com leituras desnecessárias ou transmissão de dados não diretamente relacionados com as aplicações.

Ao evitar a atualização por variações que não impactem na capacidade de \textit{containers}, a quantidade de dados transmitidos é reduzida e, consequentemente, o número de execuções das \textit{threads} \textit{Watcher} que monitoram os \textit{zNodes} também o são. Caso esta política não fosse utilizada, correria-se o risco de todos os nós escravos apresentarem variações em todas as leituras. Sabe-se que o sistema operacional possui alterações naturais dos recursos e que estas, geralmente, não representam uma grande quantidade de recursos. Além disso, segundo a política de alocação de \textit{containers} do Hadoop e utilizando os valores \textit{default}, tanto um nó com 1024 Mb livres quanto um nó com 2047 Mb livres serão capazes de inicializar somente 1 \textit{container} \cite{tg}. Por este motivo, a atualização de variações que não alterem a quantidade máxima de \textit{containers} em um nó somente desperdiçariam tráfego na rede e processamento nos nós. Em um cluster de 100 nós, por exemplo, este comportamento poderia causar a execução de 100 \textit{threads} no nó mestre a cada 30 segundos, sem levar em consideração a transmissão de dados pela rede. Contudo, uma vez que a variação representa uma alteração na capacidade de \textit{containers} do nó, a atualização deve ser feita. Caso a variação seja positiva (mais \textit{containers} podem ser lançados), a não atualização impactaria no desperdício de capacidade que ficaria ociosa. No caso da variação ser negativa (menos \textit{containers} podem ser lançados), a não atualização faria o escalonador ignorar a sobrecarga e continuar lançando novos \textit{containers} mesmo que estes estivessem acima da capacidade do nó.

%-------------------------------------------------------------------
\section{Solução Implementada}
\label{sec:solucao}
Estudos prévios indicam que o Capacity Scheduler já apresenta uma boa base para introdução de sensibilidade ao contexto e adaptatividade no Hadoop \cite{tg}. Uma vez que o objetivo deste estudo foi melhorar a adaptabilidade à ambientes com presença de compartilhamento, um dos principais mecanismos necessários foi o de adaptação à fatores que sofrem alterações no decorrer do processamento de tarefas. Assim, foi implementada uma nova funcionalidade no escalonador que permite a ele adaptar-se às variações de recursos disponíveis nos nós durante a execução de aplicações MapReduce. A solução implementada utiliza tanto os coletores de contexto quanto a comunicação distribuída por meio do ZooKeeper.

Como já citado anteriormente, o custo de aquisição e manutenção de um \textit{cluster} dedicado para a execução de aplicações MapReduce é alto e é possível que pequenas empresas, que não tenham recursos financeiros suficientes para a aquisição de um \textit{cluster}, prestem serviços que geram uma grande quantidade de dados. Embora estes casos possam ser resolvidos com a utilização de \textit{IaaS} (Infrastructure as a Service -- Infra-estrutura como Serviço) como o Amazon AWS \cite{amazonAWS} e Amazon EC2\cite{amazonEC2} a custos baixos, os proprietários podem preferir não utilizar serviços em nuvem devido à outros fatores como a segurança de dados pessoais ou adaptação à nova tecnologia. Nestes casos a utilização da capacidade ociosa dos computadores na rede da empresa pode ser uma alternativa viável, uma vez que não gera custos de aquisição e a manutenção é mais simples.

O problema da utilização do Hadoop nestas condições é que, embora em menor escala, esta alternativa reproduz muitas das características presentes em \textit{grids} pervasivos, as quais podem impactar negativamente o desempenho do Hadoop. Dentro de um contexto de disponibilidade de recursos, é possível identificar 5 situações que podem ocorrer: a saída de um nó do \textit{cluster}, a entrada de um nó no \textit{cluster}, um nó que estava sendo utilizado pelo Hadoop passa a ser utilizado em conjunto para outra aplicação (início do compartilhamento), um nó que estava sendo compartilhado volta a ser totalmente disponível para o Hadoop (fim do compartilhamento) e os computadores podem ter configurações diferentes entre si (nós heterogêneos).

A situação mais fácil de ser resolvida é a de nós heterogêneos, pois, mesmo sem alterações no comportamento do Hadoop, bastaria alterar os arquivos XML de configuração dos nós. No caso de não alteração dos arquivos XML, alguns nós poderiam ser sobrecarregados enquanto outros poderiam ser sub-utilizados. Contudo, com a solução proposta neste estudo, a disponibilidade de recursos é coletada e enviada ao escalonador. Com isso a informação estará sempre consistente com a realidade, e haverá um ganho de desempenho através da diminuição da sobrecarga e da sub-utilização.

A saída de um nó do \textit{cluster} já é suportada de maneira eficiente pelos procedimentos de tolerância à falhas do Hadoop. As tarefas que estavam sendo processadas pelo nó serão perdidas e terão que ser reiniciadas em outro nó. Além disso o \textit{pool} de recursos totais é reduzido de acordo com o tamanho do Node Manager.

A entrada (registro) de um novo nó necessita de mais etapas, mas é possível de ser realizada mesmo com a distribuição \textit{default} do Hadoop. É necessário que o \textit{host} do novo nó seja incluído no arquivo de configuração \textit{slaves} do mestre antes de ser inicializado o Node Manager no nó escravo. Uma vez que estas duas etapas estejam completas, o escalonador irá aumentar o \textit{pool} de recursos totais de acordo com a capacidade do novo nó ou 8 Gb de memória e 8 \textit{cores} na distribuição \textit{default}.

As duas situações restantes são mais interessantes, pois não há suporte para elas na distribuição \textit{default} do Hadoop. Esta capacidade de adaptação não seria necessária, uma vez que, desde seu projeto, todos os computadores que compõem o \textit{cluster} são considerados dedicados para uso do Hadoop. Ainda assim, haveria uma alternativa para a utilização compartilhada dos recursos com a distribuição \textit{default} do Hadoop. Esta alternativa seria mediante a configuração dos arquivos XML, ao configurar cada computador do \textit{cluster} com menos recursos do que os totais que ele possui. Esta alternativa, embora simples, não é eficiente, pois implica que ou a parcela não utilizada pelo Hadoop (a diferença entre o valor no arquivo XML e o valor real dos recursos) é constante, ou que haverão momentos tanto de sobrecarga como de sub-utilização do \textit{cluster}. Como já mencionado na Seção \ref{sec:zookeeper}, outra alternativa, também ineficiente, seria inicializar o \textit{cluster} com a capacidade ociosa de cada computador naquele momento configurada nos arquivos XML, executar uma aplicação MapReduce e em seguida terminar a execução dos serviços do Hadoop. Esta alternativa também está sujeita a alterações na disponibilidade de recursos e possível sobrecarga e/ou sub-utilização.

No caso de início de compartilhamento, o escalonador continua inicializando novos \textit{containers} mesmo que a capacidade do nó tenha sido excedida, causando sobrecarga e lentidão para o término das tarefas e, consequentemente, da aplicação. Com a contribuição deste estudo os recursos disponíveis para utilização pelo Hadoop no nó em questão serão reduzidos, porém nenhum \textit{container} já inicializado será preemptado. Ainda assim, a contribuição permite ao escalonador saber que determinado nó está sobrecarregado e, com base nesta informação, não alocar novos \textit{containers} neste nó até que a situação dos recursos esteja normalizada, ou seja, existirem recursos disponíveis capazes de alocar ao menos 1 \textit{container}.

No caso de término do compartilhamento a distribuição \textit{default}, quando configurada para um \textit{cluster} dedicado, terá o mesmo comportamento da solução implementada por este estudo. Como o escalonador nunca alterou a informação do nó, ao terminar o compartilhamento este nó voltará à situação inicial, a qual todo recurso do nó está disponível para o Hadoop. Nesta situação a abordagem \textit{default} do Hadoop, quando configurada para um \textit{cluster} dedicado é eficiente. No caso da solução implementada neste estudo, o nó fará a coleta, atualizará a informação no \textit{zNode} e o escalonador atualizará a capacidade do nó quando for notificado da alteração, porém haverá um atraso de no máximo 30 segundos para que as informações fiquem consistentes com relação à distribuição \textit{default}. No caso de utilização da distribuição \textit{default} configurada para utilizar somente uma parte de alguns nós estes recursos ficarão ociosos, piorando o desempenho por não utilizar todos os recursos disponíveis.
\chapter{Próximas Etapas}
%-------------------------------------------------------------------
As próximas etapas deste projeto são as seguintes:
\begin{enumerate}
   \item Decisão das métricas do algoritmo de escalonamento com base nos recursos identificados e suportados pelo \emph{Hadoop}
   \item Implementação do escalonador
   \item Testes e avaliação
   \item Ajustes e novos testes se necessário
\end{enumerate}


\setlength{\baselineskip}{\baselineskip}

%%=============================================================================
%% Referências
%%=============================================================================
\bibliographystyle{abnt}
\bibliography{referencias/referencias}



%IMPORTANTE: Se precisar usar alguma seção ou subseção dentro dos apêndices ou
%anexos, utilizar o comando \tocless para não adicionar no Sumário
%Exemplos: 
% \tocless\section{Histórico}
%%=============================================================================
%% Apêndices
%%=============================================================================
\appendix
\renewcommand{\appendixsname}{Appendix}
\chapter{Generated ResourceManager Graphs}
\label{chap:ApendA}

Following there are a brief explanation and the figures that were generated through the second method employed on the section \ref{sec:grafo}.

The pertinence of these figures is validated by the fact that they describe all possible ways a given interface can take. The perfect case flow would be started with the submission of a job to the RM. There are some pre requisites that need to be fulfilled for the start of the job. From this point on, these figures are relevant.

Firstly an AppAttempt is created. The AppAttempt is literally a started application, through which the RM will try to allocate necessary resources (Node and Containers). If the resources are successfully allocated, the real App will be created. Then an ApplicationMaster will be launched in order to manage each Application allocated RMContainers and to which RMNode they belong.

\begin{figure}[hbtn]
   \centering
   \renewcommand{\figurename}{Figure}
   \includegraphics[width=12cm]{figuras/Figura05-RMNode.png}
   \caption{RMNode's state machine}
   \label{fig:RMNode}
\end{figure}

\begin{figure}[hbtn]
   \centering
   \renewcommand{\figurename}{Figure}
   \includegraphics[width=15cm]{figuras/Figura03-RMContainer.png}
   \caption{RMContainer's state machine}
   \label{fig:RMContainer}
\end{figure}

\begin{figure}[hbtn]
   \centering
   \renewcommand{\figurename}{Figure}
   \includegraphics[width=15cm]{figuras/Figura02-RMAppAttempt.png}
   \caption{RMAppAttempt's state machine}
   \label{fig:RMAppAttempt}
\end{figure}

\begin{figure}[hbtn]
   \centering
   \renewcommand{\figurename}{Figure}
   \includegraphics[width=15cm]{figuras/Figura04-RMApp.png}
   \caption{RMApp's state machine}
   \label{fig:RMApp}
\end{figure}
%\include{capitulos/apendiceb}

%%=============================================================================
%% Anexos
%%=============================================================================
%\annex
%
\chapter{Questionários aplicados para a Análise de Tarefas}



%
\chapter{Resultados da Análise de Casos de Uso}
\begin{figure}[hbtn]
   \centering
   \includegraphics[width=6.5cm]{figuras/figura09-anexob.eps}
   \caption{Análise dos pincipais Portais da área de Informática de instituições federias}
   \label{fig:CASOSDEUSO}
\end{figure}

\end{document}
