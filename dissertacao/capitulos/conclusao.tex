\chapter{Conclusão}
%-------------------------------------------------------------------

No contexto atual do processamento distribuído de dados, uma alternativa que sempre deve ser considerada é a utilização do MapReduce e seus \textit{frameworks} de processamento. Sendo o Hadoop o mais conhecido destes, a sua utilização para o processamento de dados de maneira distribuída vem ganhando mais e mais adeptos. Embora existam serviços que disponibilizem um ambiente na nuvem para que o usuário processe seus dados, algumas pequenas empresas podem preferir outra alternativa à passar seus dados para a nuvem. Nestes casos, pode ser mais simples utilizar os recursos ociosos que estão disponíveis dentro da própria empresa.

Ainda que a utilização dos recursos ociosos da própria empresa pareça mais simples, o Hadoop não é capaz de gerenciar recursos compartilhados em um nó do \textit{cluster}. Este estudo teve como objetivo tornar o Hadoop capaz de suportar compartilhamento de recursos através de um escalonamento adaptativo com relação aos recursos disponíveis. A solução realiza esta tarefa através da coleta e transmissão de dados, em conjunto com a utilização do escalonador Capacity Scheduler. Foram realizados experimentos que apresentavam a degradação de recursos, sendo que esta degradação poderia ocorrer tanto antes do início quanto no decorrer da aplicação de MapReduce. Estes dois cenários foram enfatizados por três fatores: (1) são os cenários que ocorrem mais comumente em ambientes compartilhados; (2) a entrada/saída de nós já é suportado, com algumas peculiaridades, pelo Hadoop; e (3) a reintegração de recursos compartilhados já têm suporte com a solução implementada, porém, comparada à degradação de recursos, não possui impacto significativo no desempenho.

A solução deste estudo foi capaz de melhorar o desempenho nos casos testados com utilização compartilhada do \textit{cluster}. Os resultados alcançados neste estudo indicam que a utilização do Hadoop em um \textit{cluster} compartilhado é possível, mesmo quando este for composto por computadores pertencentes à uma pequena rede empresarial que pode, a qualquer momento, ter a utilização concorrente dos recursos em determinados nós. Como resultado do desenvolvimento deste estudo, apresentou-se uma solução, simples e não intrusiva, para a utilização do Hadoop em ambientes compartilhados. Foram utilizados apenas os recursos que inicialmente são considerados pelo Hadoop (memória e \textit{cores})  para tornar o escalonamento adaptável.

O benefício da utilização da solução deste estudo é decorrente da eliminação ou, nos piores casos, minimização da sobrecarga dos nós em virtude do compartilhamento. O simples fato de inibir o lançamento de novos \textit{containers} quando alguns nós do \textit{cluster} estão sobrecarregados apresentou um ganho de desempenho de até 40\% em alguns casos quando comparado com a distribuição \textit{default} do Hadoop. Mesmo que o ganho de desempenho apresentado seja bom, foi identificada uma falha na solução. Sendo assim, é possível que o ganho de desempenho real seja ainda maior.

Utilizando este estudo como ponto de partida, diversos trabalhos futuros são possíveis. Destacando-se, é claro, um melhor controle dos recursos sobrecarregados para que a falha discutida nos experimentos em escala seja solucionada. Além disso, é possível explorar o Application Master com objetivo de compreender e tornar o gerenciamento interno de cada aplicação adaptativo.