%\listfiles
\documentclass[diss]{mdtufsm}
% um tipo específico de monografia pode ser informado como parâmetro opcional:
%\documentclass[tese]{mdtufsm}
% a opção `openright' pode ser usada para forçar inícios de capítulos
% em páginas ímpares
% \documentclass[openright]{mdtufsm}
% para gerar uma versão frente-e-verso, use a opção 'twoside':
% \documentclass[twoside]{mdtufsm}

\usepackage[T1]{fontenc}        % pacote para conj. de caracteres correto
\usepackage{fix-cm} %para funcionar corretamente o tamanho das fontes da capa
\usepackage{times, color, xcolor}       % pacote para usar fonte Adobe Times e cores
\usepackage[utf8]{inputenc}   % pacote para acentuação
\usepackage{graphicx}  % pacote para importar figuras
\usepackage{amsmath,latexsym,amssymb} %Pacotes matemáticos
\usepackage{listings}
\usepackage{url}
\usepackage[%hidelinks%, 
            bookmarksopen=true,linktoc=none,colorlinks=true,
            linkcolor=black,citecolor=black,filecolor=magenta,urlcolor=blue,
            pdftitle={Desenvolvimento de um escalonador sensível ao contexto para o Apache Hadoop},
            pdfauthor={Guilherme Weigert Cassales},
            pdfsubject={Trabalho de Graduação},
            pdfkeywords={Apache Hadoop, escalonador, sensível ao contexto, Informática, UFSM}
            ]{hyperref} %hidelinks disponível no pacote hyperref a partir da versão 2011-02-05  6.82a
%Nesse caso, hidelinks retira os retângulos em volta dos links das referências

%Margens conforme MDT 7ª edição, arrumar diretamente no mdtufsm.cls para funcionar a opção twoside *PENDENTE*
\usepackage[inner=30mm,outer=20mm,top=30mm,bottom=20mm]{geometry} 
\usepackage{epstopdf}
\usepackage{graphicx}


%==============================================================================
% Se o pacote hyperref foi carregado a linha abaixo corrige um bug na hora
% de montar o sumário da lista de figuras e tabelas
% Se o paco1te não foi carregado, comentar a linha %
%==============================================================================
\input{macros/bugcaption}

%==============================================================================
% Identificação do trabalho
%==============================================================================
\title{Escalonamento Adaptativo para o Apache Hadoop}

\author{Cassales}{Guilherme Weigert}

\course{Curso de Ciência da Computação}
\altcourse{Curso de Ciência da Computação}

\institute{Centro de Tecnologia}
\degree{Mestre em Ciência da Computação}

% Número do TG (verificar na secretaria do curso)
% Para mestrado deixar sem opção dentro do {}
\trabalhoNumero{}

%Orientador
\advisor[]{Profª. Drª.}{Charão}{Andrea Schwertner}
%Se for uma ``orientadora'' descomentar a linha baixo
\orientadoratrue

%Avaliadores (Banca)
\committee[Prof. Dr.]{Stein}{Benhur de Oliveira}{UFSM}
\committee[Profª. Drª.]{Barcelos}{Patrícia Pitthan de Araújo}{UFSM}


% a data deve ser a da defesa; se nao especificada, são gerados
% mes e ano correntes
\date{20}{Janeiro}{2016}

%Palavras chave
\keyword{Apache Hadoop} 
\keyword{Escalonador}
\keyword{Sensibilidade ao Contexto}

%%=============================================================================
%% Início do documento
%%=============================================================================
\begin{document}

%%=============================================================================
%% Capa e folha de rosto
%%=============================================================================
\maketitle

%%=============================================================================
%% Catalogação (obrigatório para mestrado) e Folha de aprovação
%%=============================================================================
%Somente obrigatório para dissertação, para TG, remover as linhas	77	%
%Como a CIP vai ser impressa atrás da página de rosto, as margens inner e outer	
%devem ser invertidas.
\newgeometry{inner=20mm,outer=30mm,top=30mm,bottom=20mm}	
\makeCIP{cassales@inf.ufsm.br} %email do autor		
\restoregeometry

%Se for usar a catalogação gerada pelo gerador do site da biblioteca comentar as linhas
%acima e utilizar o comando abaixo
%\includeCIP{CIP.pdf}

%folha de aprovação
\makeapprove


%%=============================================================================
%% Agradecimentos (opcional)
%%=============================================================================
%\chapter*{Agradecimentos}
%Agradeço à minha família por todo apoio, não só durante os longos anos de graduação, mas em todas as etapas de minha vida.
%
%À minha namorada Raíssa, não só pelo apoio em mais um TG e mais uma graduação, mas por todo crescimento conjunto que tivemos durante um ano e quase noventa meses.
%
%Aos amigos mais próximos, que mesmo estando fisicamente distantes ou que a rotina tenha impedido um convívio diário, por sempre estarem dispostos a compartilhar experiências.
%
%À professora Andrea Charão, por todo suporte prestado tanto no papel de orientadora como de coordenadora do curso.
%
%Agradeço também, a todos que tornaram possível e/ou participaram na realização deste trabalho.


%%=============================================================================
%% Resumo
%%=============================================================================
\begin{abstract}
Hoje em dia, o volume de dados gerados é muito maior do que a capacidade de processamento dos computadores. Como solução para esse problema, algumas tarefas podem ser paralelizadas ou distribuidas. O \emph{framework Apache Hadoop} \cite{Hadoop}, é uma delas e poupa o programador as terefas de gerenciamento, como tolerância à falhas, particionamento dos dados entre outros.
Um problema no escalonador do \emph{Apache Hadoop} é que seu foco é em ambientes homogêneos, o que muitas vezes não é possível de se manter. O foco deste trabalho foi na melhora de um escalonador já existente, possuindo como objetivo torná-lo sensível ao contexto, permitindo que as capacidades físicas de cada máquina sejam consideradas na hora da distribuição das tarefas submetidas. Optou-se por inserir coletores de informações de contexto (memória e cpu) no  CapacityScheduler, tornando o comportamento desse sensível ao contexto. Através das mudanças feitas e de experimentos feitos usando um benchmark bem conhecido (TeraSort), foi possível demonstrar uma melhora no escalonamento em relação ao escalonador original com a configuração padrão.
\end{abstract}


%%=============================================================================
%% Abstract
%%=============================================================================
% resumo em inglês

\begin{englishabstract}
{Development of a context-aware scheduler for Apache Hadoop}
{Undergraduate Program in Computer Science}
{Apache Hadoop. Scheduler. Context-aware}
{January}
{th}
Nowadays the volume of data generated by the services provided for end users, is way larger than the processing capacity of one computer alone. As a solution to this problem, some tasks can be parallelized. The Apache Hadoop framework, is one of these parallelized solutions and it spares the programmer of management tasks such as fault tolerance, data partitioning, among others.
One problem on this framework is the scheduler, which is designed for homogeneous environments. It is worth to remember that maintaining a homogeneous environment is somewhat difficult today, given the fast development of new, cheaper and more powerful hardware. This work focuses on altering the Capacity Scheduler, in order to make it more context-aware towards resources on the cluster. Making it possible to consider the the physical capacities of the machines when scheduling the submitted tasks. It was chosen to insert context information (memory and cpu) collectors on CapacityScheduler, making his scheduling more context-aware. Through the changes and experiments made using a common and well known benchmark (TeraSort), it was possible to notice a improvement on scheduling in relation to the original scheduler using the default configuration.
\end{englishabstract}

%% Lista de Ilustrações (opc)
%% Lista de Símbolos (opc)
%% Lista de Anexos e Apêndices (opc)

%%=============================================================================
%% Lista de figuras (comentar se não houver)
%%=============================================================================
%\renewcommand{\listfigurename}{Figure List}
\listoffigures


%%=============================================================================
%% Lista de tabelas (comentar se não houver)
%%=============================================================================
%%\listoftables

%%=============================================================================
%% Lista de Apêndices (comentar se não houver)
%%=============================================================================
%\renewcommand{\listappendixname}{Appendix List}
%\listofappendix

%%=============================================================================
%% Lista de Anexos (comentar se não houver)
%%=============================================================================
%\listofannex

%%=============================================================================
%% Lista de abreviaturas e siglas
%%=============================================================================
 %o parametro deve ser a abreviatura mais longa
%\begin{listofabbrv}{UbiComp}
%   \item [BNF] \textit{Backus-Naur Form}
%   \item [UbiComp] Computação Ubíqua
%\end{listofabbrv}


%%=============================================================================
%% Lista de simbolos (opcional)
%%=============================================================================
%Simbolos devem aparecer conforme a ordem em que aparecem no texto
% o parametro deve ser o símbolo mais longo
%\begin{listofsymbols}{teste}
%  \item [$\varnothing$] vazio
%  \item [$\Gamma$]  Gama
%  \item [$\forall$] Para todo
%\end{listofsymbols}

%%=============================================================================
%% Sumário
%%=============================================================================
%\renewcommand{\contentsname}{Index}
%\renewcommand{\appendixsname}{Appendix}
\tableofcontents

%%=============================================================================
%% Início da dissertação
%%=============================================================================
\setlength{\baselineskip}{1.5\baselineskip}

%Adiciona cada capitulo
\chapter{Introduction}
\label{chap:Introduction}
One of the major IT companies nowadays, known as Google \cite{Google}, had the initial idea of a way to process a huge data volume generated by its servers. This approach would later be known as MapReduce, built by two separate steps, Map and Reduce, each step based on a functional language function. At the same time, a Yahoo! \cite{Yahoo} led project was starting the implementation of MapReduce for its own system, which would then become a whole new project, named Apache Hadoop \cite{Hadoop}.

Today the Apache Hadoop framework has a very active community of both developers and users, however there are some characteristics that weren't changed from the day the framework was first designed. Among these characteristics there is one very detrimental and prone to bad performance issues, which is the focus on homogeneous environments. It is known that maintaining a totally homogeneous environment is harder and harder as the time passes, requiring either a huge initial investment or a huge effort in order to replace faulty hardware without changing the component capability.

The MapReduce's task performance inside Hadoop is tightly tied to the scheduler \cite{CASH}. Since it is an open source project, it is possible to change the scheduler aiming to make it capable of better adapting to heterogeneity while at the same time presenting a performance improvement.

A key characteristic in Hadoop's transition to heterogeneous environments is the context-aware capability. The definition of context can vary from one application to another, but as a rule of thumb it is some information that the application can use as base for decision making. When an application is context-aware, it will detect and adapt to the changes in the environment \cite{Manuele}.

In the present work, the context to which the application will have to adapt is related to the physical configuration of the machines that compose the Hadoop cluster, allowing the scheduler to work with real data collected from the machines and not suppositions as the default Hadoop configuration implies. In a more complex degree, the present work is just a part of a bigger project called PER-MARE \cite{PER-MARE}, which has the objective of adapting to more environment variations, as the insertion and removal of nodes in real time.


%-------------------------------------------------------------------
\section{Objective}

The main objective of this work is to improve Hadoop through a context-aware scheduling, which will provide better performance and adaptation on heterogeneous environments.

%-------------------------------------------------------------------
\section{Motivation}

Today, some processing tasks that used to be made through huge mainframes and servers are gradually transitioning to big clusters, which are composed of computers with more accessible prices and easily bought in the market. 

Even though the Apache Hadoop framework has clusters as its target, it was designed and implemented under a specific assumption. The framework's better performance is achieved when it is running on a homogeneous cluster, in other words, when all nodes have the same resources. The problem is that given today's hardware development, it might take the system to a point where it is not possible or at least not profitable to maintain cluster homogeneity. 

Since the default configuration of the scheduler tells that every node on the cluster has the same amount of resources, if a more powerful node is inserted all that extra capacity will be wasted. This happens because the cluster will not collect the real configuration, but the parameter set in a XML file as the node capacity. The opposite is also troublesome, if a less powerful node is inserted, the cluster will use it as if it had more potential, possibly overloading that node with more tasks that it can handle and causing errors or performance issues.

The present work is relevant, as it's objective is based on adaptation and improvement of an already existent technology. With the improved scheduler, not only will the Apache Hadoop clusters have a possibility to improve cluster's resource utilization, as the framework itself will be better prepared and capable of adapting to new heterogeneous environments in an easier and smoother way.
\chapter{Fundamentos e Revisão de Literatura}

%-------------------------------------------------------------------
Este capítulo destina-se à definição de conceitos teóricos sobre as ferramentas e paradigmas utilizados no trabalho, os quais são listados a seguir: \emph{Framework Apache Hadoop}, \emph{MapReduce}, bem como trabalhos relacionados.

%-------------------------------------------------------------------
\section{Hadoop}
A origem do \emph{framework Apache Hadoop}, vem de outro projeto da \emph{Apache} \cite{Apache}, o \emph{Apache Nutch} \cite{Nutch}, que era um motor de buscas na \emph{web} com código livre iniciado em 2002. Porém o projeto encontrava problemas devido a sua arquitetura. Em 2003 quando a \emph{Google} publicou um artigo descrevendo a arquitetura utilizado no seu sistema de arquivos distribuídos, chamado GFS, os desenvolvedores viram que uma arquitetura semelhante resolveria o problema de escalabilidade do \emph{Nutch}.

Em 2004 os desenvolvedores do \emph{Nutch} começaram a implementar a ideia e o resultado foi nomeado \emph{Nutch Distributed Filesystem} (NDFS). A medida que o projeto avançava ele foi tomando proporções cada vez maiores, até que em 2006 foi criado um novo projeto pois os avanços já ultrapassavam o propósito do \emph{Nutch}, o novo projeto foi nomeado \emph{Hadoop}. O \emph{framework Hadoop} tem o propósito de facilitar o processamento distribuído através do paradigma do \emph{MapReduce}.

\subsection{Arquitetura geral do \emph{Apache Hadoop}}
De maneira geral é possível separar o \emph{Apache Hadoop} em duas partes, as quais são denominadas \emph{HDFS (Hadoop Distributed File System)} e \emph{YARN (Yet Another Resource Negotiator)}. A Figura \ref{fig:ArqGeral} fornece uma visão de como o \emph{framework} é estruturado.

\begin{figure}[hbtn]
   \centering
   \includegraphics[width=9cm]{figuras/Figura08-HadoooArchGeral.png}
   \caption{Arquitetura geral do \emph{Apache Hadoop}}
   \label{fig:ArqGeral}
\end{figure}

O HDFS é a parte responsável pelo armazenamento dos dados necessários para que os jobs sejam executados, em outras palavras, é um grande HD distribuído como indica sua denominação (\emph{Distributed File System}). O HDFS é o componente que irá fazer a replicação de tolerância a falhas, distribuição dos dados de acordo com o que cada nó irá processar, entre outras atribuições.

A outra metade do \emph{Apache Hadoop}, o YARN, é responsável pelo processamento dos \emph{jobs} submetidos ao \emph{cluster}. É dentro do YARN que as tarefas de \emph{MapReduce} são executadas, consequentemente o YARN é o componente que gerencia todos os recursos do \emph{cluster}. 

\subsubsection{HDFS}
O HDFS é em grande parte responsável pelo bom desempenho do \emph{Apache Hadoop}, pois é encarregado com a tarefa de não sobrecarregar a rede com transferência de arquivos. No HDFS o acesso a arquivo é sempre local, isso quer dizer que cada nó receberá a parte do arquivo referente a sua carga de trabalho, evitando assim replicação desnecessária além da básica para segurança e tolerância a falhas.
Um problema dessa abordagem é que o \emph{Hadoop} possui uma latência muito alta, sendo desaconselhável o uso do \emph{Hadoop} em aplicações críticas ou de tempo real. O HDFS pode ser subdivido em dois serviços, \emph{NameNode} e \emph{DataNode}, responsáveis pelo gerenciamento dos dados a nível de \emph{cluster} e gerenciamento dos dados a nível local, respectivamente. A Figura \ref{fig:ArqHDFS} apresenta um esquema básico da arquitetura do HDFS.

\begin{figure}[hbtn]
   \centering
   \includegraphics[width=8cm]{figuras/Figura07-HDFS.png}
   \caption{Arquitetura geral do HDFS \cite{HDFS}}
   \label{fig:ArqHDFS}
\end{figure}

\subsubsection{YARN}
O YARN é a parte do \emph{Apache Hadoop} responsável pela execução do \emph{MapReduce}, portanto à ele cabem as tarefas de gerenciamento e execução do processamento. Ao tornar a tarefa de processamento totalmente independente das tarefas de armazenamento, o \emph{Apache Hadoop} abre muitas possibilidades para sua utilização. Assim como o HDFS, o YARN pode ser subdividido em 2 serviços, \emph{ResourceManager} e \emph{NodeManager}, responsáveis pelo gerenciamento dos recursos no sistema e pelo gerenciamento dos recursos locais, respectivamente. A Figura \ref{fig:ArqYARN} apresenta um esquema básico da arquitetura do YARN.

\begin{figure}[hbtn]
   \centering
   \includegraphics[width=12cm]{figuras/Figura06-YarnArch.png}
   \caption{Arquitetura geral do YARN \cite{YARN}}
   \label{fig:ArqYARN}
\end{figure}

Embora não demonstrado na imagem, cada serviço possui diversos módulos internos, por exemplo o \emph{ResourceManager} possui o Escalonador e o \emph{ApplicationsManager}, os quais ainda podem ser divididos em sub-módulos menores. O presente trabalho busca apresentar uma nova solução para a maneira de escalonamento empregada no \emph{Hadoop} que se adapte melhor ao ambiente.

\subsection{Configuração do ambiente de execução do \emph{Hadoop}}
Um ambiente corretamente configurado do \emph{Hadoop} possui alguns pré-requisitos além dos nós acessíveis entre si em uma rede. Cada nó deve ter em sua instalação do \emph{Hadoop} vários arquivos xml, que são responsáveis pela configuração das MVs do \emph{Hadoop} naquela máquina.
Para conhecimento, esses arquivos são: \emph{core-site.xml, yarn-site.xml, mapred-site.xml} e \emph{hdfs-site.xml}. Cada um desses arquivos terá a configuração de um serviço do \emph{Hadoop}. O arquivo \emph{hdfs-site.xml} por exemplo é responsável pela configuração do HDFS naquela máquina.
É importante salientar que esta configuração em nenhum momento é realizada automaticamente pelo \emph{Hadoop} e que o usuário deve configurar cada nó separadamente.

%%incluir apendices/anexos com as configurações utilizadas?

\subsection{\emph{MapReduce}}
O paradigma de \emph{MapReduce}, já citado várias vezes no presente trabalho, divide o processamento em duas etapas. Essas duas etapas são derivadas das funções \emph{Map} e \emph{Reduce} das linguagens funcionais, e assim como nas funções oiginais, elas tem funcionamento baseado em tuplas de chave e valor. Um ciclo de aplicação típico é a função \emph{Map} receber um arquivo de entrada e buscar os valores procurados pela aplicação, então formar tuplas de chave e valor. Feitas as tuplas, a função \emph{Map} manda o resultado para a \emph{Reduce} onde as chaves serão processadas e reduzidas a dados mais significativos. A grande vantagem do \emph{Hadoop} é que dado um ambiente corretamente configurado, o programador pode focar sua atenção à resolução das tarefas pelo paradigma do \emph{MapReduce} e não em como o trabalho será distribuído.

\section{Sensibilidade ao contexto}
Dada a interligação dos sistemas hoje em dia, já é possível notar alguma sensibilidade ao contexto na maioria deles. Ao acessar um site por um dispositivo móvel, o site automaticamente irá carregar sua versão \emph{mobile}, a qual foi projetada para estes dispositivos, ou quando os navegadores utilizam dados de localidade para oferecer produtos, entre outros exemplos de utilização do contexto.

Segundo \cite{Zakaria}, sensibilidade ao contexto na computação se refere a habilidade de uma aplicação de detectar e responder as mudanças no ambiente de execução. O que leva a seguinte definição feita por \cite{Baldauf}, onde ele afirma que um sistema sensível ao contexto é capaz de adaptar suas operações ao contexto atual sem intervenção explicita do usuário e portanto aumentar sua usabilidade e eficácia.

Partindo dessas duas afirmações vem a dúvida sobre o que seria o contexto, portanto a definição do contexto é fundamental para que haja um entendimento da sensibilidade ao contexto \cite{Manuele}. O contexto pode assumir diversos significados dependendo da situação que se encontra, \cite{Dey} define contexto como qualquer informação que pode ser utilizada para caracterizar a situação de uma entidade (pessoa, lugar ou objeto) considerado relevante para a interação entre usuário e aplicação.

Geralmente informações de contexto são utilizadas para a melhoria de performance de um sistema ou algoritmo, portanto estima-se ser possível melhorar a execução do \emph{Apache Hadoop} através da utilização dessa técnica. 

Embora existam diversas maneiras de se utilizar essas informações de contexto para melhorar a performance do \emph{MapReduce}, \cite{Manuele} cita três exemplos de como isso pode ser feito, os quais se resumem em: configuração automática dos nós durante a instalação, gerenciamento de entrada e saída de nós do \emph{cluster} e finalmente na distribuição de tarefas feita pelo escalonador de acordo com a disponibilidade de recursos e tarefas já em execução. A terceira maneira apresentada é a maneira que corresponde à abordagem utilizada nesse trabalho.

\section{Escalonadores para Hadoop}
Uma dos principais componentes do \emph{Hadoop} é o escalonador, componente responsável pela distribuição do trabalho no ambiente. Além dos escalonadores disponibilizados juntamente com o próprio \emph{Hadoop}, existem outras implementações que buscam solucionar uma necessidade específica que os escalonadores padrões não oferecem suporte.

\subsection{\emph{Hadoop Internal Scheduler}}
O escalonador padrao do \emph{Hadoop} foi implementado visando suportar apenas a submissão de tarefas em lote. Nesse escalonador, a primeira tarefa recebida é a primeira executada, formando uma fila para as subsequentes. Apesar de simples este escalonador também suporta cinco níveis de prioridade, porém a escolha da próxima tarefa nunca deixará o tempo de submissão completamente de fora.

\subsection{\emph{Fair Scheduler}}
Utilizado para computar tarefas pequenas em lote que possuam os mesmos dados de entrada, utilizando um escalonamento em dois níveis para distribuir recursos igualitáriamente \cite{FairScheduler}. O nível superior, geralmente aloca filas para cada usuário, utilizando um algoritmo justo com pesos. O segundo nível aloca os recursos dentro de cada fila, e utiliza um algoritmo igual ao \emph{Internal Scheduler}.

\subsection{\emph{Capacity Scheduler}}
Este escalonador surgiu para os casos onde um ambiente \emph{Hadoop} é dividido entre várias empresas ou possui partes distribuídas em diversos locais sob responsabilidade de mais de um dono. Ele é focado em garantias de que uma quantidade mínima de recursos será disponibilizada a qualquer momento que um de seus usuários decidir utilizar o \emph{Hadoop}. O benefício decorre que organizações diferentes possuem picos de processamento em horas diferentes, portanto as organizações que estão utilizando o \emph{Hadoop} irão se aproveitar da capacidade ociosa das outras.

\section{Trabalhos relacionados}

Foi feita uma pesquisa bibliográfica com objetivo de analisar os trabalhos que já haviam sido desenvolvidos envolvendo o \emph{Hadoop} e que se propunham a alterar ou adaptar o escalonador. Além disso, buscou-se identificar quais técnicas eram as mais utilizadas e em cima de quais objetivos o trabalho foi desenvolvido. A seguir encontram-se os trabalhos relacionados e um breve resumo sobre a proposta, contexto utilizado e objetivo esperado com as alterações.

\begin{itemize}
\item CASH (\emph{Context Aware Scheduler for Hadoop}) \cite{CASH}, nesse trabalho o objetivo dos autores é de melhorar o rendimento geral do \emph{cluster}. Eles partem da hipótese de que grande parte dos jobs são periódicos e executados no mesmo horário, além de possuírem características de uso de CPU, rede, disco etc. semelhantes. O trabalho ainda leva em consideração que com o passar do tempo os nós tendem a ficar mais heterogêneos. Com a intenção de solucionar esses problemas e baseados nessas hipóteses, foi implementado um escalonador que classifica tanto os jobs como as máquinas com relação ao seu potencial de CPU e E/S, podendo então distribuir os jobs para máquinas que tem uma configuração apropriada para sua natureza.

\item LATE (\emph{Longest Approximation Time to End}) \cite{LATE}, seguindo o que o nome sugere, nesse tabalho a informação de contexto é referente ao tempo estimado de término da \emph{task} baseado numa heurística que faz a relação de tempo decorrido e \emph{score}. Essa informação é usada também para gerar um limiar de quando uma \emph{task} é lenta o suficiente para indicar sitomas de erros e então iniciar uma nova em outra máquina possívelmente mais rápida. O objetivo do trabalho era de reduzir o tempo de resposta em \emph{clusters} grandes que executam muitos \emph{jobs} de pequena duração.

\item \emph{A Dynamic MapReduce Scheduler for Heterogeneous Workloads} \cite{DMRSHW}, aqui os autores também utilizam a técnica de classificar os \emph{jobs} e máquinas de acordo com a quantidade de E/S ou CPU. E assim como no CASH, o principal objetivo é a melhora de rendimento no \emph{cluster}. Uma das diferenças, no entanto, é que essa implementação utiliza um escalonador com três filas.

\item SAMR (\emph{A Self-adaptative MapReduce}) \cite{SAMR}, essa implementação segue a mesma ideia do LATE, onde a informação de contexto é referente ao cálculo do progresso de uma \emph{task} para identificar se é necessário lançar outra task igual ou não. Porém essa solução varia o cálculo do progresso de acordo com informações do ambiente em que a \emph{task} está sendo executada, seu principal objetivo é a redução do tempo de execução das \emph{tasks}. Para que sejam utilizadas informações do ambiente, o algoritmo leva em consideração informações históricas contidas em cada nó, e ajusta o peso de cada estágio do processamento.

\item COSHH (\emph{A Classification and Optimization based Scheduler for Heterogeneous Hadoop Systems}) \cite{COSHH}, um pouco mais abrangente das demais soluções apresentadas, essa solução leva em consideração informações não especificadas sobre o sistema. Seu ganho de performance se dá a partir da classificação dos \emph{jobs} em classes, ele então faz uma busca por máquinas que se encaixem nessa mesma classe. Essa busca é feita por um algoritmo que reduz o tamanho do espaço de busca para melhorar o rendimento. O objetivo dessa solução é a melhora do tempo médio em que os \emph{jobs} são completados, além de oferecer uma boa performance quando utilizando somente o fatia mínima, além de proporcionar uma distribuição justa.

\item \emph{Quincy} \cite{Quincy}, diferentemente de todos os outros trabalhos, essa solução não foi desenvolvida visando somente o \emph{Hadoop} mas ainda assim é aplicável ao mesmo. Possuindo como objetivo melhorar o desempenho geral de um \emph{cluster}, utiliza como informação de contexto a distribuição de recursos e modifica a maneira tradicional de tratamento desses. Ao utilizar um modo mais dinâmico do que o convencional, a solução mapeia os recursos num grafo de capacidades e demandas e calcula o escalonamento ótimo a partir de uma função global de custo.

\item \emph{Improving MapReduce Performance through Data Placement in heterogeneous Hadoop Clusters} \cite{IMRPDPHHC}, buscando melhorar a performance de \emph{jobs} que possuam muito processamento de dados através da melhor distribuição desses dados, essa solução utiliza principalmente a localidade dos dados como informação para tomada de decisões. O ganho de performance é dado pelo rebalanceamento dos dados nos nós, deixando nós mais rápidos com mais dados. Isso diminui o custo de \emph{jobs} especulativos e de transferência de dados pela rede.
\end{itemize}

Após estudo dos trabalhos, nota-se que muitos deles tem por objetivo a diminuição do tempo de resposta ou a melhoria do rendimento de maneira geral, os quais diferem dos objetivos do presente trabalho, que na verdade busca proporcionar uma melhor adaptação do \emph{Hadoop} a um ambiente heterogêneo. 

Constata-se também que há uma diversidade de contextos levados em consideração, contudo é possível identificar temas recorrentes como : a classificação dos \emph{jobs} e dos nós quanto ao potencial de E/S ou de CPU, a avaliação do progresso da \emph{task} na decisão de lançar ou não uma nova \emph{task} especulativa.
\chapter{Métodos e Desenvolvimento}
\label{cap:desen}
Este capítulo descreve as etapas de desenvolvimento e as metodologias empregadas neste trabalho. Buscaram-se estratégias sem intrusão ou grandes modificações nas políticas de escalonamento já implementadas pelo \textit{framework}. Nas Seções \ref{sec:collector} e \ref{sec:zookeeper} são apresentados em maior detalhe os coletores de contexto e a ferramenta de comunicação distribuída utilizados neste trabalho, respectivamente. Finalmente, na Seção \ref{sec:solucao} faz-se uma análise em profundidade da solução implementada comparando-a com o comportamento \textit{default} do Hadoop em ambientes compartilhados.

%-------------------------------------------------------------------
\section{Coletores de Contexto}
\label{sec:collector}
Após um estudo aprofundado dos escalonadores do Hadoop, ficou claro que o Capacity Scheduler já está estruturado de maneira a oferecer escalabilidade e um desempenho satisfatório. Porém, este escalonador possui um ponto fraco, o qual é exposto quando utilizado em um ambiente compartilhado. O Capacity Scheduler só recebe informações sobre os Node Managers no momento da inicialização destes, e ainda, a informação é obtida de um arquivo de configuração. A importância de que as informações sobre os Node Managers estejam atualizadas decorre da dependência intrínseca do escalonamento com a disponibilidade de recursos, na qual uma informação errada pode influenciar o algoritmo de maneira prejudicial e diminuir o desempenho das tarefas. Com base nestas observações, o primeiro passo para a solução é a coleta de dados sobre os recursos dos Node Managers, ou seja, a adição de sensibilidade ao contexto.

\subsection{Sensibilidade ao Contexto}
\label{sec:ctx}
Contexto é definido por \cite{Dey} como qualquer informação que pode ser utilizada para caracterizar a situação de uma entidade (pessoa, lugar ou objeto) considerada relevante para a interação entre usuário e aplicação. Um exemplo de utilização de contexto é quando um usuário acessa um site por meio de um dispositivo móvel e o site carrega automaticamente a versão \textit{mobile}, a qual possui alterações que aumentam a compatibilidade com o tipo de dispositivo sendo utilizado. Outra situação semelhante ocorre quando dados de localidade são utilizados para melhorar os resultados de um motor de buscas, mostrando primeiro os resultados mais relacionados com a região ou idioma do usuário. Nota-se que nas duas situações a aplicação utiliza dados específicos, o tipo do dispositivo e a localização do usuário, coletados no momento da execução e utiliza-os para adaptar o seu funcionamento de maneira a oferecer maior conforto ou usabilidade ao usuário. 

Sendo assim, se uma aplicação é capaz de coletar informações sobre a situação do sistema no qual está sendo executada, ela é capaz de coletar informações de contexto. Porém a simples coleta não aumenta, de fato, o desempenho desta aplicação, é necessário que a aplicação seja capaz de responder às mudanças detectadas no ambiente. Esta capacidade de detecção e reação é caracterizada como sensibilidade ao contexto por \cite{Maamar} e vai ao encontro da definição de \cite{Baldauf} em que o sistema deve detectar mudanças e adaptar suas operações sem intervenção explícita do usuário, aumentando assim a usabilidade e eficácia da aplicação.

\subsection{Implementação}
Em um primeiro momento, buscou-se diminuir a dependência nos arquivos XML para a configuração dos recursos dos Node Managers com intuito de facilitar a configuração inicial dos nós e futuramente utilizar este mesmo mecanismo para a inclusão do suporte ao compartilhamento dos recursos. Para isto, fez-se necessário o desenvolvimento de um conjunto de coletores de contexto capazes de coletar de maneira eficiente os recursos do nó em questão no momento da inicialização do serviço Node Manager e, consequentemente, passar a informação correta sobre os recursos no momento do registro no Resource Manager.

Embora esta adição já facilite a utilização do Hadoop em um \textit{cluster} de natureza heterogênea, ainda não é suficiente para que o Hadoop seja utilizado eficientemente em um \textit{cluster} compartilhado. A razão para esta afirmação é que, embora a informação esteja correta no momento de inicialização, é possível que, durante a execução das aplicações de \textit{MapReduce}, os recursos comecem a ser utilizados por outros usuários, diminuindo a capacidade disponível para o Hadoop. O problema causado pela utilização de alguns nós por outros usuários sem a devida atualização dos dados no escalonador é a sobrecarga destes nós, uma vez que o Hadoop irá tentar utilizar memória e processadores que já estão alocados à outros processos.
% causando, nos piores casos, \textit{Swap} da memória.
%TODO Se deixar isso, colocar dados de swap nos testes

Com objetivo de solucionar o problema causado pelo compartilhamento, é necessário que a coleta de dados ocorra não apenas na inicialização do Node Manager mas ao longo da execução do serviço em intervalos periódicos. Para isso optou-se pela utilização de uma \textit{thread} para a coleta e transmissão, se necessária, dos dados.

O coletor escolhido para a tarefa foi o coletor desenvolvido pelo projeto PER-MARE \cite{Collector}, o qual utiliza a interface padrão do Java para monitoramento, OperatingSystemMXBean \cite{MXBean}. A implementação deste coletor de contexto é baseada em uma interface, uma classe abstrata e as classes de coleta dos recursos desejados. Devido ao seu projeto, coletores de novas informações podem ser facilmente criados, aumentando assim a quantidade de informação disponível para o escalonador.

A interface OperatingSystemMXBean, possibilita o acesso às informações do sistema no qual a JVM está sendo executada. Uma vez que a classe abstrata implementa esta interface, todas suas herdeiras poderão utilizá-la.

As classes utilizadas neste trabalho fazem a coleta de memória física disponível e processadores disponíveis, os recursos suportados por padrão no Hadoop. É possível visualizar o digrama de classes na Figura \ref{fig:collectorUML}, onde estão presentes alguns exemplos de possíveis coletores a serem utilizados.

%TODO fazer outro DC
\begin{figure}[!hbtn]
   \centering
   \includegraphics[width=15cm]{figuras/CollectorUML2.pdf}
   \caption{Diagrama de classes dos coletores de contexto}
   \label{fig:collectorUML}
\end{figure}

Cada instância de Node Manager possui um conjunto de coletores (um para memória e um para processadores), os quais realizam a coleta num intervalo pré-definido. Os coletores são executados por uma \textit{thread} independente e possuem um intervalo de 30 segundos entre as coletas para não causar sobrecarga ou interrupção no  processamento das tarefas \textit{Map} e \textit{Reduce}.

%-------------------------------------------------------------------
\section{Comunicação Distribuída}
\label{sec:zookeeper}

Para que a informação coletada pelos coletores de contexto possa afetar o escalonamento é necessário que exista uma maneira para os Node Managers transmitirem os dados atualizados ao escalonador, a ferramenta escolhida para esta tarefa foi o ZooKeeper.

\subsection{ZooKeeper}
O ZooKeeper é um projeto da Apache e fornece ferramentas eficientes, confiáveis e tolerantes à falha para a coordenação de sistemas distribuídos \cite{Hunt2010}. Inicialmente, o ZooKeeper foi implementado como um componente do Hadoop e virou um projeto próprio conforme cresciam suas funcionalidades e sua utilização em outras aplicações. 

A arquitetura utilizada no ZooKeeper é a de cliente-servidor, sendo o servidor o próprio ZooKeeper (chamado de \textit{ensamble}), enquanto a aplicação que o está utilizando assume o papel de cliente. Os dados do ZooKeeper ficam armazenados em \textit{zNodes}, abstrações que podem ser tanto um \textit{container} de dados quanto de outros \textit{zNodes}, e formam um sistema de arquivos hierárquico que pode ser comparado à estrutura de uma árvore. Para garantir a consistência deste sistema de arquivos o ZooKeeper utiliza operações de escrita linearizáveis, as quais são obrigatoriamente processadas pelo servidor líder que é, então, encarregado de propagar as mudanças para os demais participantes do \textit{ensamble} \cite{Pham}.

Um dos recursos do ZooKeeper que oferece grande utilidade é o \textit{Watcher}, uma interface que permite aos clientes o monitoramento de certos \textit{zNodes}. Quando um \textit{Watcher} é registrado como monitor de um \textit{zNode}, ele pode ser configurado para monitorar a alteração dos dados do \textit{zNode}, a criação/remoção de \textit{zNodes} filhos, ou ainda para qualquer tipo de alteração no \textit{zNode} e seus filhos. Quando um \textit{zNode} sofre uma alteração que o \textit{Watcher} está monitorando uma \textit{callback} é disparada para que o cliente faça o processamento desejado. O disparo desta \textit{callback} é um evento único, forçando o programador a re-inserir um \textit{Watcher} no \textit{zNode} para continuar monitorando-o \cite{HadoopBook}.

No contexto deste trabalho, os serviços do ZooKeeper são utilizados para monitorar as informações de contexto coletadas nos nós escravos e transmiti-las para o escalonador. Sendo que a comunicação é feita através de processos que atualizam e monitoram o conteúdo de \textit{zNodes}.

%----------------------
\subsection{Implementação}
A flexibilidade oferecida através dos \textit{zNodes} permite que qualquer estrutura de dado seja inserida como informação, porém existem alguns detalhes importantes no gerenciamento de \textit{Watchers} que podem impactar na eficácia da solução. Por esta razão optou-se por utilizar um \textit{zNode} para cada Node Manager, o que permite realizar um controle mais rígido das atualizações e possibilita uma maneira fácil e rápida tanto para a inserção das informações por parte dos Node Managers quanto para o monitoramento dos dados pelo escalonador.

Na solução adotada cada \textit{zNode} contém informação de apenas um Node Manager e para cada \textit{zNode} existe um Watcher, sendo que cada \textit{Watcher} é uma thread pertencente ao escalonador. Qualquer alteração em um \textit{zNode} dispara uma \textit{callback} para seu \textit{Watcher} específico, o qual lê os novos dados do \textit{zNode} e atualiza a informação no escalonador sobre os recursos daquele nó. A utilização do \textit{Watcher} permite que o escalonador realize operações apenas quando necessário e não desperdice tempo percorrendo os \textit{zNodes} quando não houverem atualizações a serem feitas. A estrutura adotada no trabalho podes ser visualizada na Figura \ref{fig:zk}.


%Qualquer alteração na tabela dispara uma \textit{callback} no escalonador, que por sua vez percorre a tabela à procura das modificações e realiza as alterações pertinentes nas informações sobre os recursos. Esta abordagem permite que o escalonador realize operações apenas quando necessário e não desperdice tempo acessando a tabela quando não houverem atualizações a fazer. A estrutura adotada no trabalho podes ser visualizada na Figura \ref{fig:zk}.


\begin{figure}[!hbtn]
   \centering
   \includegraphics[width=15cm]{figuras/Zookeeper.pdf}
   \caption{Estrutura do ZooKeeper}
   \label{fig:zk}
\end{figure}

A solução escolhida gera dois papéis para os nós, o papel de monitoramento e o papel de atualização, os quais são explicados em detalhe a seguir.

O papel de \textbf{Monitoramento} é desempenhado pelo escalonador, o qual possui uma \textit{thread} \textit{Watcher} que monitora um \textit{zNode}. Este \textit{zNode} inicial servirá como \textit{container} para os demais \textit{zNodes}, os quais armazenarão as informações de contexto referentes aos nós escravos do \textit{cluster}. O \textit{Watcher} do \textit{zNode} inicial permite ao escalonador ser notificado quando novos \textit{zNodes} forem inseridos na estrutura. Quando a \textit{callback} do \textit{Watcher} é acionada, a \textit{thread} recebe todos os \textit{zNodes} filhos daquele que está sendo monitorado e após identificar o novo \textit{zNode}, inicia uma nova \textit{thread Watcher} para monitorar o novo \textit{zNode} identificado. Desta forma, cada \textit{zNode} será monitorado por uma \textit{thread Watcher} única. Um possível problema da técnica utilizada é discutido, juntamente com a solução adotada para solucioná-lo, na descrição do papel de Atualização.

Uma alternativa à utilização de um \textit{zNode} para cada nó escravo, seria a utilização de uma estrutura que permitisse a inserção de todos os dados dos nós escravos, como uma Tabela Hash. Embora, em um primeiro momento, o maior impecilho para a utilização desta técnica possa parecer o limite padrão do tamanho da informação que um \textit{zNode} comporta (1 Mb), o problema é, na verdade, relacionado com a utilização de uma única \textit{thread Watcher}. 

Como todos os nós escravos são, geralmente, inicializados ao mesmo tempo, as coletas de dado são realizadas em períodos semelhantes, estes períodos serão denominados neste estudo como fase de coleta. Durante estas fases, é possível que muitas atualizações sejam feitas em um pequeno espaço de tempo, gerando um risco à consistência das informações, uma vez que o ZooKeeper não fornece garantias de que todas as alterações notificarão a \textit{thread Watcher}. Este comportamento ocorre devido à possibilidade de notificações serem disparadas antes da re-inserção do \textit{Watcher} no \textit{zNode} que contém a Tabela Hash. Esta falha só ocorre quando a segunda notificação é enviada antes que a primeira termine de ser processada, e existem duas situações que isto pode ocorrer. A primeira situação não apresenta grandes riscos à consistência das informações, pois a notificação perdida  ocorre entre duas notificações normais na mesma fase de coleta. Sendo assim, a notificação seguinte fará com que o escalonador ajuste os valores de recursos dos dois nós. Já a segunda situação apresenta riscos à consistência das informações, pois ocorre quando não há uma notificação normal na mesma fase de coleta após a notificação perdida, ou seja, a notificação perdida só terá suas informações atualizadas no escalonador na próxima fase de coleta, a qual ocorrerá em aproximadamente 30 segundos. Este período que o escalonador opera com informações inconsistentes pode causar o lançamento de \textit{containers} que irão exceder o limite dos nós caso um novo usuário tenha iniciado a utilização do nó, gerando sobrecarga e lentidão tanto na aplicação do novo usuário como na aplicação MapReduce.

%	o papel de monitor é desempenhado pelo escalonador. O escalonador implementa a interface Watcher, fornecida pelo ZooKeeper, e monitora o \textit{zNode} que contém a Tabela Hash. No momento da inicialização o escalonador cria um \textit{zNode} e insere uma Tabela Hash vazia como informação, então finalmente, inicia o monitoramento do \textit{zNode}. Quando a \textit{callback} é acionada, a \textit{thread} percorre a tabela em busca de qual informação foi atualizada, atualiza os dados do escalonador e reinicia o monitoramento.
	
O papel de \textbf{Atualização} é realizado pelos Node Managers, os quais lançam, no momento de sua inicialização, uma \textit{thread} responsável por fazer a coleta de dados e, caso houver alteração com relação à ultima coleta, atualizar os dados no \textit{zNode}. A coleta dos dados é realizada a cada 30 segundos, intervalo que corresponde à média de tempo observada em \textit{containers} executados em um \textit{cluster} de funcionamento normal. Com objetivo de evitar o envio de dados por alterações naturais do sistema e que não impactariam no escalonamento foi implementada uma política de atualização, na qual atualizações só serão realizadas quando as variações alterarem o limite de \textit{containers} que podem ser inicializados no nó, seja pela carga de CPU ou disponibilidade de memória. Esta política permite ao escalonador otimizar seu poder de adaptabilidade sem desperdiçar tempo com leituras desnecessárias ou transmissão de dados não diretamente relacionados com as aplicações.

Ao evitar a atualização por variações que não impactem na capacidade de \textit{containers}, a quantidade de dados transmitidos é reduzida e, consequentemente, o número de execuções das \textit{threads} \textit{Watcher} que monitoram os \textit{zNodes} também o são. Caso esta política não fosse utilizada, correria-se o risco de todos os nós escravos apresentarem variações em todas as leituras. Sabe-se que o sistema operacional possui alterações naturais dos recursos e que estas, geralmente, não representam uma grande quantidade de recursos. Além disso, segundo a política de alocação de \textit{containers} do Hadoop e utilizando os valores \textit{default}, tanto um nó com 1024 Mb livres quanto um nó com 2047 Mb livres serão capazes de inicializar somente 1 \textit{container} \cite{tg}. Por este motivo, a atualização de variações que não alterem a quantidade máxima de \textit{containers} em um nó somente desperdiçariam tráfego na rede e processamento nos nós. Em um cluster de 100 nós, por exemplo, este comportamento poderia causar a execução de 100 \textit{threads} no nó mestre a cada 30 segundos, sem levar em consideração a transmissão de dados pela rede. Contudo, uma vez que a variação representa uma alteração na capacidade de \textit{containers} do nó, a atualização deve ser feita. Caso a variação seja positiva (mais \textit{containers} podem ser lançados), a não atualização impactaria no desperdício de capacidade que ficaria ociosa. No caso da variação ser negativa (menos \textit{containers} podem ser lançados), a não atualização faria o escalonador ignorar a sobrecarga e continuar lançando novos \textit{containers} mesmo que estes estivessem acima da capacidade do nó.

%-------------------------------------------------------------------
\section{Solução Implementada}
\label{sec:solucao}
Estudos prévios indicam que o Capacity Scheduler já apresenta uma boa base para introdução de sensibilidade ao contexto e adaptatividade no Hadoop \cite{tg}. Uma vez que o objetivo deste estudo foi melhorar a adaptabilidade à ambientes com presença de compartilhamento, um dos principais mecanismos necessários foi o de adaptação à fatores que sofrem alterações no decorrer do processamento de tarefas. Assim, foi implementada uma nova funcionalidade no escalonador que permite a ele adaptar-se às variações de recursos disponíveis nos nós durante a execução de aplicações MapReduce. A solução implementada utiliza tanto os coletores de contexto quanto a comunicação distribuída por meio do ZooKeeper.

Como já citado anteriormente, o custo de aquisição e manutenção de um \textit{cluster} dedicado para a execução de aplicações MapReduce é alto e é possível que pequenas empresas, que não tenham recursos financeiros suficientes para a aquisição de um \textit{cluster}, prestem serviços que geram uma grande quantidade de dados. Embora estes casos possam ser resolvidos com a utilização de \textit{IaaS} (Infrastructure as a Service -- Infra-estrutura como Serviço) como o Amazon AWS \cite{amazonAWS} e Amazon EC2\cite{amazonEC2} a custos baixos, os proprietários podem preferir não utilizar serviços em nuvem devido à outros fatores como a segurança de dados pessoais ou adaptação à nova tecnologia. Nestes casos a utilização da capacidade ociosa dos computadores na rede da empresa pode ser uma alternativa viável, uma vez que não gera custos de aquisição e a manutenção é mais simples.

O problema da utilização do Hadoop nestas condições é que, embora em menor escala, esta alternativa reproduz muitas das características presentes em \textit{grids} pervasivos, as quais podem impactar negativamente o desempenho do Hadoop. Dentro de um contexto de disponibilidade de recursos, é possível identificar 5 situações que podem ocorrer: a saída de um nó do \textit{cluster}, a entrada de um nó no \textit{cluster}, um nó que estava sendo utilizado pelo Hadoop passa a ser utilizado em conjunto para outra aplicação (início do compartilhamento), um nó que estava sendo compartilhado volta a ser totalmente disponível para o Hadoop (fim do compartilhamento) e os computadores podem ter configurações diferentes entre si (nós heterogêneos).

A situação mais fácil de ser resolvida é a de nós heterogêneos, pois, mesmo sem alterações no comportamento do Hadoop, bastaria alterar os arquivos XML de configuração dos nós. No caso de não alteração dos arquivos XML, alguns nós poderiam ser sobrecarregados enquanto outros poderiam ser sub-utilizados. Contudo, com a solução proposta neste estudo, a disponibilidade de recursos é coletada e enviada ao escalonador. Com isso a informação estará sempre consistente com a realidade, e haverá um ganho de desempenho através da diminuição da sobrecarga e da sub-utilização.

A saída de um nó do \textit{cluster} já é suportada de maneira eficiente pelos procedimentos de tolerância à falhas do Hadoop. As tarefas que estavam sendo processadas pelo nó serão perdidas e terão que ser reiniciadas em outro nó. Além disso o \textit{pool} de recursos totais é reduzido de acordo com o tamanho do Node Manager.

A entrada (registro) de um novo nó necessita de mais etapas, mas é possível de ser realizada mesmo com a distribuição \textit{default} do Hadoop. É necessário que o \textit{host} do novo nó seja incluído no arquivo de configuração \textit{slaves} do mestre antes de ser inicializado o Node Manager no nó escravo. Uma vez que estas duas etapas estejam completas, o escalonador irá aumentar o \textit{pool} de recursos totais de acordo com a capacidade do novo nó ou 8 Gb de memória e 8 \textit{cores} na distribuição \textit{default}.

As duas situações restantes são mais interessantes, pois não há suporte para elas na distribuição \textit{default} do Hadoop. Esta capacidade de adaptação não seria necessária, uma vez que, desde seu projeto, todos os computadores que compõem o \textit{cluster} são considerados dedicados para uso do Hadoop. Ainda assim, haveria uma alternativa para a utilização compartilhada dos recursos com a distribuição \textit{default} do Hadoop. Esta alternativa seria mediante a configuração dos arquivos XML, ao configurar cada computador do \textit{cluster} com menos recursos do que os totais que ele possui. Esta alternativa, embora simples, não é eficiente, pois implica que ou a parcela não utilizada pelo Hadoop (a diferença entre o valor no arquivo XML e o valor real dos recursos) é constante, ou que haverão momentos tanto de sobrecarga como de sub-utilização do \textit{cluster}. Como já mencionado na Seção \ref{sec:zookeeper}, outra alternativa, também ineficiente, seria inicializar o \textit{cluster} com a capacidade ociosa de cada computador naquele momento configurada nos arquivos XML, executar uma aplicação MapReduce e em seguida terminar a execução dos serviços do Hadoop. Esta alternativa também está sujeita a alterações na disponibilidade de recursos e possível sobrecarga e/ou sub-utilização.

No caso de início de compartilhamento, o escalonador continua inicializando novos \textit{containers} mesmo que a capacidade do nó tenha sido excedida, causando sobrecarga e lentidão para o término das tarefas e, consequentemente, da aplicação. Com a contribuição deste estudo os recursos disponíveis para utilização pelo Hadoop no nó em questão serão reduzidos, porém nenhum \textit{container} já inicializado será preemptado. Ainda assim, a contribuição permite ao escalonador saber que determinado nó está sobrecarregado e, com base nesta informação, não alocar novos \textit{containers} neste nó até que a situação dos recursos esteja normalizada, ou seja, existirem recursos disponíveis capazes de alocar ao menos 1 \textit{container}.

No caso de término do compartilhamento a distribuição \textit{default}, quando configurada para um \textit{cluster} dedicado, terá o mesmo comportamento da solução implementada por este estudo. Como o escalonador nunca alterou a informação do nó, ao terminar o compartilhamento este nó voltará à situação inicial, a qual todo recurso do nó está disponível para o Hadoop. Nesta situação a abordagem \textit{default} do Hadoop, quando configurada para um \textit{cluster} dedicado é eficiente. No caso da solução implementada neste estudo, o nó fará a coleta, atualizará a informação no \textit{zNode} e o escalonador atualizará a capacidade do nó quando for notificado da alteração, porém haverá um atraso de no máximo 30 segundos para que as informações fiquem consistentes com relação à distribuição \textit{default}. No caso de utilização da distribuição \textit{default} configurada para utilizar somente uma parte de alguns nós estes recursos ficarão ociosos, piorando o desempenho por não utilizar todos os recursos disponíveis.
\chapter{Experimentos e Resultados}
\label{chap:ExpRes}
%-------------------------------------------------------------------
Esta capítulo contém informações detalhadas sobre os experimentos realizados e os resultados obtidos. Embora o principal caso estudado foi o de degradação dos recursos em virtude de um compartilhamento dos nós, os experimentos realizados podem ser divididos em três categorias de acordo com seu objetivo. As três categorias são: (1) experimentos realizados em ambiente manipulado para obtenção de indícios preliminares da usabilidade da solução; (2) experimentos realizados utilizando a implementação real para otenção de comprovação da eficácia da solução; (3) experimentos em escala para obtenção de informações com relação à escalabilidade da solução.

\section{Considerações Iniciais Sobre os Experimentos}
\label{sec:casosteste}
Em virtude de algumas informações sobre os experimentos serem comuns à todos eles, como os casos de teste e as aplicações utilizadas, esta Seção é destinada à apresentação destas informações. Além disso, a apresentação dos resultados utiliza diagramas de Gantt com uma estrutura modificada em relação à usualmente encontrada na literatura e será, também, explicada detalhadamente nesta Seção.

A primeira informação relevante que refere-se à todos experimentos são os casos de teste. Todos os experimentos utilizam os mesmos casos de teste, possuindo apenas objetivos diferentes. Os casos de teste representam situações de compartilhamento, sendo que 2 casos utilizam a implementação padrão e 2 casos utilizam a solução proposta por este trabalho. Além disso, com a criação dos casos de teste buscou-se facilitar a comparação entre os resultados alcançados. A descrição dos casos de testes encontra-se a seguir.

\textbf{Caso A:} representa uma situação sem compartilhamento, onde o usuário possui acesso à todos os recursos do \textit{cluster} em qualquer momento. Isto implica que os recursos informados ao escalonador \textbf{sempre} corresponderão aos recursos disponíveis para o Hadoop. Consideram-se recursos informados como os dados que o escalonador utiliza para realizar suas políticas de escalonamento, enquanto, recursos disponíveis são aqueles estão livres e/ou sendo utilizados pelo próprio Hadoop. Utilizando uma notação percentual, os recursos informados são de 100\% e os recursos disponíveis são de 100\% durante toda execução.

\textbf{Caso B:} representa a situação decorrente do compartilhamento dos nós do \textit{cluster} com outros usuários. Como consequência do compartilhamento, é possível que em, algum momento, ocorra uma inconsistência entre a quantidade de recursos informada e disponível. Este caso aplica o comportamento padrão do Hadoop, no qual os recursos são informados por meio de arquivos XML \textbf{somente} na inicialização do serviço e nunca são atualizados. Em notação percentual, os recursos informados são de 100\%, porém os recursos disponíveis são de 50\%.

\textbf{Caso C:} repete as especificações do Caso B, porém possui a implementação descrita no Capítulo \ref{cap:desen}. Este caso representa a situação de quando outra aplicação é lançada \textbf{antes} da ocorrência da coleta e transmissão de dados, ou seja, quando uma nova aplicação for submetida ao \textit{cluster}, este já estará com os dados atualizados. Em notação percentual, os recursos informados são de 50\% e os recursos disponíveis são de 50\%.

\textbf{Caso D:} representa uma extensão do Caso C em que a inicialização de outra aplicação ocorre \textbf{após} a coleta e transmissão dos dados e \textbf{antes} da submissão de uma aplicação, ou seja, a aplicação será lançada numa situação onde o \textit{cluster} possui a informação errada (Caso B) e terá de se adaptar à nova configuração dos recursos (Caso C) durante a execução. Em notação percentual, os recursos informados no início da aplicação são de 100\%, enquanto os recursos disponíveis são de 50\%. Após a coleta e transmissão de dados os recursos informados também passam a ser 50\%.

Além dos casos de teste, outra característica importante e comum à todos experimentos são as aplicações. Embora aplicações de \textit{Big Data} geralmente possuem dependência de memória, outros fatores como a utilização de CPU e E/S podem influenciar no desempenho. Na busca de comprovações de que a solução apresenta ganhos quando utilizada com aplicações de diferentes características, decidiu-se pela utilização de 3 aplicações de \textit{benchmark}, cada uma com diferentes requisições de memória, CPU e E/S. As aplicações são as seguintes:

\begin{itemize}
	\item TeraSort: o objetivo do TeraSort \citep{TeraSort2008} é ordenar um conjunto de dados o mais rápido possível. Este \textit{benchmark} de ordenação estressa tanto a memória como o CPU em virtude das comparações e armazenamento temporário;
	\item WordCount: o \textit{benchmark} WordCount é um exemplo básico de \textit{MapReduce}. Seu objetivo é contar o número de ocorrências de cada palavra de um texto. Como a utilização de memória e E/S é limitada nesta aplicação (tanto na etapa de processamento como a saída da aplicação possuem estruturas pequenas em comparação ao arquivo de entrada), o desempenho desta aplicação é determinado pelo CPU;
	\item TestDFSIO: o \textit{benchmark} TestDFSIO é um teste de leitura e escrita para o HDFS. Este \textit{benchmark} é útil para estressas o HDFS, descobrir \textit{bottlenecks} na rede, SO e configuração do Hadoop. O objetivo é prover uma mensuração de quão rápido o \textit{cluster} é em termos de E/S. Tanto a memória quanto o CPU são pouco utilizados.
\end{itemize}

Optou-se pela utilização das aplicações implementadas no \textit{HiBench} \cite{HiBench}, um conjunto de \textit{benchmarks} para \textit{clusters} Hadoop que foi utilizado nos trabalhos \cite{HBA} \cite{HBB} \cite{HBC}. O tamanho de entrada utilizado para cada aplicação foi: um conjunto de dados de 15 GB para o Terasort, 90 arquivos de 250 MB para o TestDFSIO e um arquivo de 10 GB para o WordCount. 

Ainda ligado aos experimentos porém com relação aos resultados, os diagramas de Gantt apresentados neste trabalho são modificados para inclusão de mais informações. A apresentação dos diagramas está agrupada por aplicação, sendo que e cada aplicação possui 4 diagramas. Cada um dos diagramas de uma aplicação corresponde a um caso de teste e todos os diagramas utilizam a mesma escala de tempo.

Cada diagrama possui 2 ou mais linhas, sendo que cada linha representa  um recurso (nó do \textit{cluster}). Estas linhas de recurso possuem diversos "separadores" verticais (formando diversos segmentos), os quais indicam que um \textit{container} iniciou/terminou sua execução. Cada segmento apresenta diferentes alturas e tons de cores, os quais são utilizados para representar a carga de \textit{containers} do nó. Quanto mais escuro for o tom de um segmento, mais \textit{containers} ele possui em execução; o mesmo aplica-se para a altura do segmento, quanto mais alto mais \textit{containers} em execução.

Embora as análises foram feitas principalmente com os \textit{containers} Map, os \textit{containers} de Reduce e do Application Master consomem recursos do \textit{cluster} e devem ser apresentados para uma representação fiel da situação real. Por este motivo, o \textit{container} Application Master é representado na cor verde, os \textit{containers} de Reduce são representados pela cor azul e os \textit{containers} Map são representados em escalas de cinza, sendo branco indicando 0 \textit{containers} e preto indicando 16 \textit{containers}.

\section{Experimento controlado}
Este experimento foi realizado com objetivo de obter indícios de que a solução poderia apresentar melhoria no processo de escalonamento quando o Hadoop é utilizado num ambiente que existe a degradação dos recursos em virtude de compartilhamento. O experimento simplifica a solução para facilitar a obtenção dos indícios em menor tempo. A situação desejada de expressar com o experimento é de quando os nós do \textit{cluster} começam a ser utilizados por outros usuários antes/durante a aplicação \textit{MapReduce}.

\subsection{Configurações de Hardware e Software}
O experimento foi realizado no \textit{cluster} genepi do Grid'5000. A configuração do \textit{cluster} utilizado no experimento foi a de 1 mestre e 4 escravos, sendo que cada um destes nós possuem a seguinte configuração: 2 CPUs Intel(R) Xeon(R) E5420 2.5GHz (totalizando 8 cores por nó) e 8 GB RAM. Todos os nós do experimento possuíam o sistema operacional Ubuntu x64-12.04, com a JDK 1.8 instalada e a versão 2.6.0 do Hadoop configurada. Todas as informações foram obtidas através do sistema de \textit{logs} do Hadoop.

\subsection{Procedimentos}
Para que o experimento fosse totalmente controlado, decidiu-se utilizar a manipulação de informações e exclusão de nós para representar o compartilhamento. Na situação real (ver Seção \ref{sec:expReal}) o \textit{cluster} teria 4 escravos e todos os escravos teriam seus recursos disponíveis reduzidos pela metade, ou seja, outra aplicação utilizaria 4 cores e 4 GB de memória em cada nó. Para representar esta situação de maneira simples e rápida, o experimento foi realizado com apenas 2 nós, com a informação sobre a quantidade de recursos dobrada, ou seja, 16 cores e 16 GB de memória. 

\subsection{Resultados e Interpretações}
Os resultados dos experimentos podem ser visualizados nas Tabelas \ref{tab:exp1TS}, \ref{tab:exp1IO} e \ref{tab:exp1WC} e nas Figuras \ref{fig:exp1TS}, \ref{fig:exp1IO} e \ref{fig:exp1WC}. Tanto as Tabelas como as Figuras estão ordenados na ordem TeraSort, TestDFSIO e WordCount. 

\begin{table}[h!]
	\caption{Resumo dos resultados do TeraSort controlado em segundos.} \label{tab:exp1TS}
	\begin{tabular*}{\hsize}{lllll} %{\hsize}{@{\extracolsep{\fill}}lllll@{}}
		%\toprule
		\textbf{Caso} & \textbf{A} & \textbf{B} & \textbf{C} & \textbf{D}\\
		\hline
		Tempo Total de Map ({\it{s}}) & 149 & 788 & 348 & 477 \\
		Tempo Médio de Map ({\it{s}}) & 39.47 & 222.97 & 38.38 & 68.42 \\
		Desvio Padrão & 15.73 & 59.86 & 18.09 & 29.91 \\
		\# Tarefas Map & 76 & 76 & 76 & 76 \\
		\# Tarefas Especulativas & 2 & 1 & 3 & 1 \\
		%\botrule
	\end{tabular*}
\end{table}

\begin{table}[h!]
	\caption{Resumo dos resultados do TestDFSIO controlado em segundos.} \label{tab:exp1IO}
	\begin{tabular*}{\hsize}{lllll} %{\hsize}{@{\extracolsep{\fill}}lllll@{}}
		%\toprule
		\textbf{Caso} & \textbf{A} & \textbf{B} & \textbf{C} & \textbf{D}\\
		\hline
		Tempo Total de Map ({\it{s}}) & 139 & 444 & 239 & 364 \\
		Tempo Médio de Map ({\it{s}}) & 38.95 & 85.01 & 32.20 & 81.62 \\
		Desvio Padrão & 17.20 & 69.08 & 8.30 & 73.60 \\
		\# Tarefas Map & 90 & 90 & 90 & 90 \\
		\# Tarefas Especulativas & 0 & 9 & 0 & 1 \\
		%\botrule
	\end{tabular*}
\end{table}


\begin{table}[h!]
	\caption{Resumo dos resultados do WordCount controlado em segundos.} \label{tab:exp1WC}
	\begin{tabular*}{\hsize}{lllll} %{\hsize}{@{\extracolsep{\fill}}lllll@{}}
		%\toprule
		\textbf{Caso} & \textbf{A} & \textbf{B} & \textbf{C} & \textbf{D}\\
		\hline
		Tempo Total de Map ({\it{s}}) & 155 & 1009 & 309 & 805 \\
		Tempo Médio de Map ({\it{s}}) & 43.76 & 208.39 & 41.73 & 175.80 \\
		Desvio Padrão & 15.61 & 128.90 & 10.99 & 151.59 \\
		\# Tarefas Map & 90 & 90 & 90 & 90 \\
		\# Tarefas Especulativas & 1 & 15 & 1 & 10 \\
		%\botrule
	\end{tabular*}
\end{table}

\begin{figure}[!ht]
	\centering
	\includegraphics[width=1\textwidth]{figuras/todos.png}
	\caption{Diagrama de Gantt para os experimentos controlados com TeraSort}
	\label{fig:exp1TS}
\end{figure}

\begin{figure}[!ht]
	\centering
	\includegraphics[width=1\textwidth]{figuras/todos-DFSIO.png}
	\caption{Diagrama de Gantt para os experimentos controlados com TestDFSIO}
	\label{fig:exp1IO}
\end{figure}

\begin{figure}[!ht]
	\centering
	\includegraphics[width=1\textwidth]{figuras/todos-WC.png}
	\caption{Diagrama de Gantt para os experimentos controlados com WordCount}
	\label{fig:exp1WC}
\end{figure}

\section{Experimento real}
\label{sec:expReal}
Este experimento foi realizado com objetivo de obter provas reais de que a solução apresenta melhoria no processo de escalonamento quando o Hadoop é utilizado num ambiente que existe a degradação dos recursos em virtude de compartilhamento. O experimento utiliza a solução descrita no Capítulo \ref{cap:desen}. A situação expressada com este experimento é de quando os nós do \textit{cluster} começam a ser utilizados por outros usuários antes/durante a aplicação \textit{MapReduce}.

\subsection{Configurações de Hardware e Software}
O experimento foi realizado no \textit{cluster} genepi do Grid'5000. A configuração do \textit{cluster} utilizado no experimento foi a de 1 mestre e 4 escravos, sendo que cada um destes nós possuem a seguinte configuração: 2 CPUs Intel(R) Xeon(R) E5420 2.5GHz (totalizando 8 cores por nó) e 8 GB RAM. Todos os nós do experimento possuíam o sistema operacional Ubuntu x64-12.04, com a JDK 1.8 instalada e a versão 2.6.0 do Hadoop configurada. Todas as informações foram obtidas através do sistema de \textit{logs} do Hadoop.

\subsection{Procedimentos}
Diferentemente do experimento anterior, neste experimento buscou-se o comportamento da solução em um ambiente realmente compartilhado. O \textit{cluster} possui 4 escravos e todos os escravos terão, em algum momento, seus recursos disponíveis reduzidos pela metade devido ao compartilhamento, ou seja, outra aplicação irá utilizar 4 cores e 4 GB de memória em cada nó. Para alcançar esta situação foi implementada uma aplicação em C, na qual a quantidade desejada de \textit{threads} é inicializada e cada uma delas utiliza 1 core e 1 GB de memória.


\subsection{Resultados e Interpretações}
The comparison of node memory from default and collector implementation can be seen in the table \ref{tab:experiments}.

\section{Experimento de escala}
Uma vez que os experimentos anteriores já responderam algumas questões importantes sobre a viabilidade da solução implementada, este experimento foi realizado com objetivo de obter provas de que a solução apresenta melhoria no processo de escalonamento mesmo quando o Hadoop é utilizado em um ambiente de grande escala com uma aplicação de grande escala. O experimento utiliza a solução descrita no Capítulo \ref{cap:desen}. A situação expressada com este experimento é de quando os nós do \textit{cluster} começam a ser utilizados por outros usuários antes/durante a aplicação \textit{MapReduce}.

\subsection{Configurações de Hardware e Software}
%TODO mudar descrição para ficar de acordo com o experimento
O experimento foi realizado no \textit{cluster} genepi do Grid'5000. A configuração do \textit{cluster} utilizado no experimento foi a de 1 mestre e 4 escravos, sendo que cada um destes nós possuem a seguinte configuração: 2 CPUs Intel(R) Xeon(R) E5420 2.5GHz (totalizando 8 cores por nó) e 8 GB RAM. Todos os nós do experimento possuíam o sistema operacional Ubuntu x64-12.04, com a JDK 1.8 instalada e a versão 2.6.0 do Hadoop configurada.

\subsection{Procedimentos}
%TODO arrumar quantidade de escravos
Neste experimento buscou-se o comportamento da solução em um ambiente realmente compartilhado e de grande escala. O \textit{cluster} possui X escravos e todos os escravos terão, em algum momento, seus recursos disponíveis reduzidos pela metade devido ao compartilhamento, ou seja, outra aplicação irá utilizar 4 cores e 4 GB de memória em cada nó. Para alcançar esta situação foi implementada uma aplicação em C, na qual a quantidade desejada de \textit{threads} é inicializada e cada uma delas utiliza 1 core e 1 GB de memória.


\subsection{Resultados e Interpretações}
The comparison of node memory from default and collector implementation can be seen in the table \ref{tab:experiments}.

%
%\section{Original CapacityScheduler X Context-aware CapacityScheduler}
%This experiment was performed in order to compare the container allocation pattern in the original CapacityScheduler and the context-aware CapacityScheduler. The experiment consisted in executing a TeraSort in the cluster with the original CapacityScheduler and the context-aware CapacityScheduler. This was made in order to compare how the higher resource availability and higher allocation limits impacted the scheduling.
%
%\subsection{Hardware and Software configuration}
%The experiment used the same hardware configuration from the previous one. Regarding the Hadoop configuration, there are new properties used. The properties are the minimum and maximum allocation values, which are set in properties stated on section \ref{sec:alloc}. The only difference being that one of the executions had the collector plugged.
%
%\subsection{Procedures}
%The procedure chosen as data acquisition method was the Hadoop Log System. The reason for such a choice was that Hadoop Log System is, by default, enabled in the INFO level and using the INFO level would be possible to insert small entries and extract useful information in real time. The data was acquired with the same call during the execution of services with both schedulers.
%
%The application used to test the scheduling was a TeraSort with 5GB data to sort, requesting enough containers and providing enough data to be processed in order to stress the cluster.
%
%
%\subsection{Results and interpretation}
%Before going further into the interpretation of the results, there are some characteristics of jobs that need to be reminded. If the number of reduce tasks parameter isn't set on \textit{mapred-site.XML}, the default value used is 1, making the whole reduce step forced to be executed on only one container.
%
%Another thing to note is that the first allocated container is always the ApplicationMaster, making this container not relevant in grant of resources for MapReduce tasks analysis. Thus both the ApplicationMaster and Reducer container were withdraw from the data analyzed, which was left only with the Map containers. All times are normalized related to the first Map container created.
%
%The cluster configuration achieved with the original CapacityScheduler was: 
%\begin{itemize}
%	\item Total cluster resource of 32768mb and 32cores
%	\item Minimum allocation of 1024mb and 1 core
%	\item Maximum allocation of 8192mb and 32 cores.
%	\item All Map containers were granted containers of 1024mb and 1 core, the minimum limit.
%\end{itemize}
%
%
%The cluster configuration achieved with the context-aware CapacityScheduler was: 
%\begin{itemize}
%	\item Total cluster resource of 193210mb and 96cores
%	\item Minimum allocation of 4830mb and 2 cores
%	\item Maximum allocation of 24151mb and 12 cores
%	\item All Map containers were granted containers of 4830mb and 2 cores, the minimum limit.
%\end{itemize}
%
%Although a huge difference was achieved by only comparing the resources collected and the allocation limits, the main objective of this work is to impact the scheduling performance in a Hadoop cluster. Therefore, a TeraSort execution was made and the results achieved are discussed below.
%
%The following Gantt Charts are consolidated by resources, which are the NodeManagers. This means that the tasks, in this case portrayed as containers, will be consolidated to the resources they are tied. As stated before, the containers are allocated to a certain NodeManager. The consolidation works in a way that when a separation occurs in the segment, it means that a container has either started or finished on that NodeManager. That implies that many containers will be on more than one segment, and, the numbers inside the segment indicates which containers are running at that moment.
%
%Figure \ref{fig:ganttDefault} portraits the Gantt Chart of the TeraSort with original CapacityScheduler. It is easy to notice that some containers had to wait for the completion of others in order to start processing their tasks.
%
%\begin{figure}[hbtn]
%   \renewcommand{\figurename}{Figure}
%   \centering
%   \includegraphics[width=16cm]{figuras/Figura16-GanttDefault.png}
%   \caption{Container assignment with default configuration}
%   \label{fig:ganttDefault}
%\end{figure}
%
%In order to illustrate how to interpret the Gantt charts, the node stremi-42 of figure \ref{fig:ganttDefault} will be taken as example. It starts all its containers, numbered from 16 to 23, at the 0 seconds mark, then the segment ends at the 21 seconds mark, meaning that either a container started or finished. After a quick analysis of the containers in the first and second segments, it is possible to note that containers 17, 18, 21 and 22 are not in the second segment, meaning they have finished processing their tasks. Another thing to notice is that on the second segment, containers with numbers 37, 38 and 39 appeared for the first time, meaning they were started at this time. If the analysis is extended to the segment from 22 to 23 seconds, it is possible to note that containers 16, 19 and 20 have finished processing their tasks too, and the only running containers in this node at this moment are the containers 23, 37, 38 and 39.
%
%Figure \ref{fig:ganttImproved} portraits the Gantt Chart of the TeraSort with context-aware CapacityScheduler. In this case the overall completion time was reduced, this happened due to the fact that all containers could be started right after the arrival of the request, thanks to the higher resource availability.
%
%\begin{figure}[hbtn]
%   \renewcommand{\figurename}{Figure}
%   \centering
%   \includegraphics[width=15cm]{figuras/Figura17-GanttImproved.png}
%   \caption{Container assignment with the improved configuration}
%   \label{fig:ganttImproved}
%\end{figure}
%
%After an analysis and comparison of both charts, it is possible to notice that the default chart has containers 41-43 started on node stremi-5 and container 44 started on node stremi-42, while the context-aware chart has only the standard containers, which are numbered 2-39. This happens because these extra containers are, in reality, speculative tasks launched because other tasks were taking to long to finish. Without a better information acquisition it is hard to determine the match of original and speculative containers, but it is possible to infer which containers would make possible candidates, leaving containers 2-9, 23, 28 and 30 as possible staggers, responsible for the launching of speculative containers 41-43. Because of the same reasons, it is only possible to infer that container 44 was launched because the most likely stagger was one container in the 32-36 range.
%
%By analysing the container numberings it is possible to notice how the scheduler decides which node is going to be used. The containers launched on a given node follow a logic numbering, meaning that the resources of that container are used until exhaustion before the scheduler starts launching containers on another node. 

%\section{Heterogeneity simulation}
%This experiment was performed in order to simulate a heterogeneous environment and test how well would the context-aware would adapt. The experiment consisted in executing a TeraSort in the cluster with the simulated heterogeneous environment using context-aware CapacityScheduler. 
%
%\subsection{Hardware and Software configuration}
%The experiment used the same hardware configuration from the previous experiments. Regarding the Hadoop configuration, there are no changes. The only difference is that the nodes are purposely given false capacities when being added to the RM. Using this false values, a heterogeneous cluster will be simulated.
%
%\subsection{Procedures}
%The procedure chosen as data acquisition method was the Hadoop Log System. The reason for such a choice was that Hadoop Log System is, by default, enabled in the INFO level and using the INFO level would be possible to insert small entries and extract useful information in real time. The data was acquired with the same call during the execution of services with both schedulers.
%
%The application used to test the scheduling was a TeraSort with 5GB data to sort, requesting enough containers and providing enough data to be processed in order to stress the cluster.
%
%\subsection{Results and interpretation}
%As this experiment is a replication of the last one plus the simulated heterogeneity, the same principles applies regarding the container analysis. 
%
%It is important to firstly know the configuration of the simulated heterogeneity. The cluster had the following simulated configuration:
%
%\begin{itemize}
%	\item stremi-17: 28981 MB of memory and 14 cores.
%	\item stremi-22: 34715 MB of memory and 18 cores.
%	\item stremi-33: 46287 MB of memory and 24 cores.
%	\item stremi-35: 24151 MB of memory and 12 cores.
%	\item Total Cluster Resources: 134134 MB of memory and 68 cores.
%	\item Minimum Allocation: 3353 MB of memory and 1 core.
%\end{itemize}
%
%Figure \ref{fig:ganttSimulation} portraits the Gantt Chart of the TeraSort execution within the simulated heterogeneous environment, also using context-aware CapacityScheduler. Compared to the default case, the heterogeneous environment execution shows an improvement, but due to the lower cluster capacity, it is a slightly worse than the context-aware CapacityScheduler executing on a homogeneous environment.
%
%\begin{figure}[hbtn]
%   \renewcommand{\figurename}{Figure}
%   \centering
%   \includegraphics[width=15cm]{figuras/Figura21-GanttSimulation.png}
%   \caption{Container assignment with the simulated heterogeneous environment}
%   \label{fig:ganttSimulation}
%\end{figure}
%
%It is possible to note that the containers started the assignment with the node stremi-33, which is the node with the most capacity in the cluster and also was the first to be added in the node list. As in the other experiments, the scheduler launches containers on a node until its resources are all reserved, then mode to the next node on the list.
%
%On this experiment a speculative task was launched. Contrary to the other experiments, its easy to infer which was the original stagger task, since there was only one container active at the moment that the container 41 was launched. It is also possible to note that the scheduler didn't change nodes to launch the speculative, that happens because the node had spare capacity when the request for the speculative arrived.
%
%This experiment shows that it is possible to use this context-aware in a heterogeneous environment, the allocations were adapted to a slightly smaller cluster if compared to the real environment. As a future work, it is possible to set the allocation limits in function not only of total cluster resources but also of each individual node resource capacity.
\chapter{Conclusão}
%-------------------------------------------------------------------

No contexto atual do processamento distribuído de dados, uma alternativa que sempre deve ser considerada é a utilização do MapReduce e seus \textit{frameworks} de processamento. Sendo o Hadoop o mais conhecido destes, a sua utilização para o processamento de dados de maneira distribuída vem ganhando mais e mais adeptos. Embora existam serviços que disponibilizem um ambiente na nuvem para que o usuário processe seus dados, algumas pequenas empresas podem preferir outra alternativa à passar seus dados para a nuvem. Nestes casos, pode ser mais simples utilizar os recursos ociosos que estão disponíveis dentro da própria empresa.

Ainda que a utilização dos recursos ociosos da própria empresa pareça mais simples, o Hadoop não é capaz de gerenciar recursos compartilhados em um nó do \textit{cluster}. Este estudo teve como objetivo tornar o Hadoop capaz de suportar compartilhamento de recursos através de um escalonamento adaptativo com relação aos recursos disponíveis. A solução realiza esta tarefa através da coleta e transmissão de dados, em conjunto com a utilização do escalonador Capacity Scheduler. Foram realizados experimentos que apresentavam a degradação de recursos, sendo que esta degradação poderia ocorrer tanto antes do início quanto no decorrer da aplicação de MapReduce. Estes dois cenários foram enfatizados por três fatores: (1) são os cenários que ocorrem mais comumente em ambientes compartilhados; (2) a entrada/saída de nós já é suportado, com algumas peculiaridades, pelo Hadoop; e (3) a reintegração de recursos compartilhados já têm suporte com a solução implementada, porém, comparada à degradação de recursos, não possui impacto significativo no desempenho.

A solução deste estudo foi capaz de melhorar o desempenho nos casos testados com utilização compartilhada do \textit{cluster}. Os resultados alcançados neste estudo indicam que a utilização do Hadoop em um \textit{cluster} compartilhado é possível, mesmo quando este for composto por computadores pertencentes à uma pequena rede empresarial que pode, a qualquer momento, ter a utilização concorrente dos recursos em determinados nós. Como resultado do desenvolvimento deste estudo, apresentou-se uma solução, simples e não intrusiva, para a utilização do Hadoop em ambientes compartilhados. Foram utilizados apenas os recursos que inicialmente são considerados pelo Hadoop (memória e \textit{cores})  para tornar o escalonamento adaptável.

O benefício da utilização da solução deste estudo é decorrente da eliminação ou, nos piores casos, minimização da sobrecarga dos nós em virtude do compartilhamento. O simples fato de inibir o lançamento de novos \textit{containers} quando alguns nós do \textit{cluster} estão sobrecarregados apresentou um ganho de desempenho de até 40\% em alguns casos quando comparado com a distribuição \textit{default} do Hadoop. Mesmo que o ganho de desempenho apresentado seja bom, foi identificada uma falha na solução. Sendo assim, é possível que o ganho de desempenho real seja ainda maior.

Utilizando este estudo como ponto de partida, diversos trabalhos futuros são possíveis. Destacando-se, é claro, um melhor controle dos recursos sobrecarregados para que a falha discutida nos experimentos em escala seja solucionada. Além disso, é possível explorar o Application Master com objetivo de compreender e tornar o gerenciamento interno de cada aplicação adaptativo.

\setlength{\baselineskip}{\baselineskip}

%%=============================================================================
%% Referências
%%=============================================================================
%\renewcommand{\bibname}{References}
\bibliographystyle{abnt}
\bibliography{referencias/referencias}

%IMPORTANTE: Se precisar usar alguma seção ou subseção dentro dos apêndices ou
%anexos, utilizar o comando \tocless para não adicionar no Sumário
%Exemplos: 
% \tocless\section{Histórico}
%%=============================================================================
%% Apêndices
%%=============================================================================
%\renewcommand{\appendixname}{Appendix}
%\appendix
%\renewcommand{\appendixsname}{Appendix}
\chapter{Generated ResourceManager Graphs}
\label{chap:ApendA}

Following there are a brief explanation and the figures that were generated through the second method employed on the section \ref{sec:grafo}.

The pertinence of these figures is validated by the fact that they describe all possible ways a given interface can take. The perfect case flow would be started with the submission of a job to the RM. There are some pre requisites that need to be fulfilled for the start of the job. From this point on, these figures are relevant.

Firstly an AppAttempt is created. The AppAttempt is literally a started application, through which the RM will try to allocate necessary resources (Node and Containers). If the resources are successfully allocated, the real App will be created. Then an ApplicationMaster will be launched in order to manage each Application allocated RMContainers and to which RMNode they belong.

\begin{figure}[hbtn]
   \centering
   \renewcommand{\figurename}{Figure}
   \includegraphics[width=12cm]{figuras/Figura05-RMNode.png}
   \caption{RMNode's state machine}
   \label{fig:RMNode}
\end{figure}

\begin{figure}[hbtn]
   \centering
   \renewcommand{\figurename}{Figure}
   \includegraphics[width=15cm]{figuras/Figura03-RMContainer.png}
   \caption{RMContainer's state machine}
   \label{fig:RMContainer}
\end{figure}

\begin{figure}[hbtn]
   \centering
   \renewcommand{\figurename}{Figure}
   \includegraphics[width=15cm]{figuras/Figura02-RMAppAttempt.png}
   \caption{RMAppAttempt's state machine}
   \label{fig:RMAppAttempt}
\end{figure}

\begin{figure}[hbtn]
   \centering
   \renewcommand{\figurename}{Figure}
   \includegraphics[width=15cm]{figuras/Figura04-RMApp.png}
   \caption{RMApp's state machine}
   \label{fig:RMApp}
\end{figure}
%\chapter{Grid'5000 execution environment configuration}

This appendix has the necessary steps in order to setup a correctly working execution environment on Grid'5000 cluster.

The Apache Hadoop's installation, contrary to end users standard program installations, doesn't have a graphical interface. Actually the installation is just the extraction of files to a determined folder, however the environment configuration isn't trivial and demands some network administration knowledge.

For the correct functioning of the framework, it is necessary to edit of some files responsible for describing the environment in which it will execute. Besides, it is necessary to have a network in which every node has access to the others. Knowing that Apache Hadoop has suffered some heavy changes in changing from 1.x to 2.x, it was expected that installing both versions would make the differences clearer.

In the beginning of this work the objective was to install and configure the Apache Hadoop in three situations:
\begin{itemize}
        \item localhost, a single-node case.
        \item mini cluster, a multi-node case.
        \item Grid'5000, another multi-node case.
\end{itemize}

There are two configuration steps, the network step which refers to the configuration that doesn't depend on Hadoop and the Hadoop step which refers to the parameters that influence Hadoop's execution. This step is composed of tasks executed on the operational system, such as the user creation and ssh configuration. The second step, on the other hand, is the proper Hadoop environment configuration and, therefore, it has to change some configuration files in a way that Hadoop can be executed without errors.

The OS configuration is equal on both versions, since it has already been noted that it is independent of Hadoop. The Hadoop configuration, however, has a few differences between versions. Even so, both versions follow the same general rules. The Hadoop configuration is made through a bunch of xml files already mentioned in section \ref{chap:fundamentacao}, subsection \ref{sec:envconfig}, in which some properties containing name and value are inserted. This properties will be responsible for changing the default Hadoop behavior, but aside some properties changing their names, the biggest difference is that YARN version has a new xml file named \textit{yarn-site.xml}, it contains all parameters related to the YARN framework and therefore has a high influence on MapReduce and job execution performance.

Initially it was chosen to configure both versions on every specified instance, however, given the project focus the version 1.x was no longer a viable alternative and it's use was discontinued.

The localhost environment configuration, is a simple process and occurred  without problems after a basic contact with the framework. The mini cluster environment, was configured using two machines, already presenting a difference, in which the master could also be a slave on 1.x, in other words, it was possible to run NameNode and DataNode on the same node. On YARN version, the NameNode machine can't start a DataNode service, and the same is valid for ResourceManager and NodeManager. This change makes the tasks of processing and management totally apart and differentiable making the cluster more organized.

The real experiment starts when the Hadoop is deployed on a real cluster, like the Grid'5000, it was decided to use only the YARN version from this point onward. The installation of YARN requires only a few changes from mini cluster to Grid'5000 environment. The Grid also supplies a lot of tools that makes the job easier for Hadoop management.

At the end of this stage, some Hadoop peculiarities have already been cleared and the execution environment for posterior testing of the implementations was already deployed. Also, it is important to note that it was possible to identify some contexts that would present a high difficult to change the behavior. One of these behaviors would be the addition or removal of nodes on real time after the cluster initialization. The difficult comes from the way the service uses static reference to xml files that are read only on the initialization, making it impossible to use their values without restarting the services.
%
\chapter{Source code edition and compilation}
This appendix has the necessary steps in order to setup a correctly working compilation environment.

Given the project nature of generating a improved, context-aware, scheduler for the Apache Hadoop, it is necessary that this scheduler is included on the final distribution. Not only it has to be included, it has to be available for use by everyone who wants to try it. In order to achieve this, new jar files have to be generated from the modified code. Because of this a previous study was made about the necessary requisites in order to compile and generate these jar files.

The study began with the official documentation consultation, which included Apache Hadoop web site and help files included on the source code distributions. This way it was possible to create some steps required to compile the code. Starting from this list it was possible to identify the required dependencies to compile the code, which were then installed. The dependencies were: JDK 1.6 or higher, Maven 3.0, ProtocolBuffer 2.4.1 or higher and Cmake2.6 or higher.

The greatest objective of this process was to discover how the compilation took place and also how it would behave with the addition of new classes to standard code. Aiming to achieve this objective, a new but simple scheduler class was added. After the compilation process ended, the generated jar files were then copied to the Grid'5000 and deployed there in order to test if it would be possible to execute the compiled version in that environment.

Once the Hadoop services were deployed, it could be proven that the new scheduler was being used. It was also possible to identify the same vulnerability in the previous stage, in which the service has to be restarted in order to modify some of the Hadoop's parameters. 
%\chapter{Example of semi-processed log from a experiment}
\label{chap:logs}

The experiment results were collected using the Log System from Hadoop. Here is an example of a log already semi-processed, which means that this log has been filtered to show only the entries relevant to the analysis.

The following log snippet, shows the log when an application was submitted, in this case the application was the TeraSort. It is possible to note a lot of information, like the user who submitted and queue used, the applicationId, among others.

\lstset{language=Java,
             basicstyle=\footnotesize,
             numbers=left,
             extendedchars=\true
             showspaces = false,
             numberstyle=\footnotesize,
             frame=shadowbox,
             breaklines = true}

\begin{lstlisting}
2014-01-13 13:19:21,517 INFO org.apache.hadoop.yarn.server.resourcemanager.scheduler.capacity.LeafQueue: Application application_1389615375211_0002 from user: hadoop activated in queue: default
2014-01-13 13:19:21,518 INFO org.apache.hadoop.yarn.server.resourcemanager.scheduler.capacity.LeafQueue: Application added - appId: application_1389615375211_0002 user: org.apache.hadoop.yarn.server.resourcemanager.scheduler.capacity.LeafQueue$User@5673e296, leaf-queue: default #user-pending-applications: 0 #user-active-applications: 1 #queue-pending-applications: 0 #queue-active-applications: 1
\end{lstlisting}

Another interesting log snippet is the one that show the assignment and completion of containers. The following snippet shows when the Reduce and ApplicationMaster containers are completed and the application is finished.

\begin{lstlisting}
2014-01-13 13:24:05,016 INFO org.apache.hadoop.yarn.server.resourcemanager.scheduler.capacity.LeafQueue: completedContainer container=Container: [ContainerId: container_1389615375211_0002_01_000040, NodeId: stremi-44.reims.grid5000.fr:34048, NodeHttpAddress: stremi-44.reims.grid5000.fr:8042, Resource: <memory:4830, vCores:1>, Priority: 10, Token: Token { kind: ContainerToken, service: 172.16.160.44:34048 }, ] resource=<memory:4830, vCores:1> queue=default: capacity=1.0, absoluteCapacity=1.0, usedResources=<memory:4830, vCores:1>usedCapacity=0.024998447, absoluteUsedCapacity=0.024998447, numApps=1, numContainers=1 usedCapacity=0.024998447 absoluteUsedCapacity=0.024998447 used=<memory:4830, vCores:1> cluster=<memory:193212, vCores:96>
2014-01-13 13:24:11,146 INFO org.apache.hadoop.yarn.server.resourcemanager.scheduler.capacity.LeafQueue: default used=<memory:0, vCores:0> numContainers=0 user=hadoop user-resources=<memory:0, vCores:0>
2014-01-13 13:24:11,147 INFO org.apache.hadoop.yarn.server.resourcemanager.scheduler.capacity.LeafQueue: completedContainer container=Container: [ContainerId: container_1389615375211_0002_01_000001, NodeId: stremi-7.reims.grid5000.fr:58215, NodeHttpAddress: stremi-7.reims.grid5000.fr:8042, Resource: <memory:4830, vCores:1>, Priority: 0, Token: Token { kind: ContainerToken, service: 172.16.160.7:58215 }, ] resource=<memory:4830, vCores:1> queue=default: capacity=1.0, absoluteCapacity=1.0, usedResources=<memory:0, vCores:0>usedCapacity=0.0, absoluteUsedCapacity=0.0, numApps=1, numContainers=0 usedCapacity=0.0 absoluteUsedCapacity=0.0 used=<memory:0, vCores:0> cluster=<memory:193212, vCores:96>
2014-01-13 13:24:11,150 INFO org.apache.hadoop.yarn.server.resourcemanager.scheduler.capacity.LeafQueue: Application removed - appId: application_1389615375211_0002 user: hadoop queue: default #user-pending-applications: 0 #user-active-applications: 0 #queue-pending-applications: 0 #queue-active-applications: 0
\end{lstlisting}

Finally, there is something that influences a lot on the results, which is the time that an action took place. As it was possible to see on the above examples, the Hadoop Log System provides the hour, minute, second and milliseconds information. Thanks to this, two assignments that happened with mere milliseconds of difference were shown as 1 second delayed on chapter \ref{chap:Experiments and Results}. The first container belongs to node stremi-44 and started at 13:19:29,995. The second container belongs to stremi-42 and started at 13:19:30:079

\begin{lstlisting}
2014-01-13 13:19:29,995 INFO org.apache.hadoop.yarn.server.resourcemanager.scheduler.capacity.LeafQueue: assignedContainer application=application_1389615375211_0002 container=Container: [ContainerId: container_1389615375211_0002_01_000030, NodeId: stremi-44.reims.grid5000.fr:34048, NodeHttpAddress: stremi-44.reims.grid5000.fr:8042, Resource: <memory:4830, vCores:1>, Priority: 20, Token: Token { kind: ContainerToken, service: 172.16.160.44:34048 }, ] containerId=container_1389615375211_0002_01_000030 queue=default: capacity=1.0, absoluteCapacity=1.0, usedResources=<memory:140070, vCores:29>usedCapacity=0.72495496, absoluteUsedCapacity=0.72495496, numApps=1, numContainers=29 usedCapacity=0.72495496 absoluteUsedCapacity=0.72495496 used=<memory:140070, vCores:29> cluster=<memory:193212, vCores:96>
2014-01-13 13:19:30,079 INFO org.apache.hadoop.yarn.server.resourcemanager.scheduler.capacity.LeafQueue: assignedContainer application=application_1389615375211_0002 container=Container: [ContainerId: container_1389615375211_0002_01_000031, NodeId: stremi-42.reims.grid5000.fr:43999, NodeHttpAddress: stremi-42.reims.grid5000.fr:8042, Resource: <memory:4830, vCores:1>, Priority: 20, Token: Token { kind: ContainerToken, service: 172.16.160.42:43999 }, ] containerId=container_1389615375211_0002_01_000031 queue=default: capacity=1.0, absoluteCapacity=1.0, usedResources=<memory:144900, vCores:30>usedCapacity=0.74995345, absoluteUsedCapacity=0.74995345, numApps=1, numContainers=30 usedCapacity=0.74995345 absoluteUsedCapacity=0.74995345 used=<memory:144900, vCores:30> cluster=<memory:193212, vCores:96>
\end{lstlisting}

%\chapter{Main code changes performed}
\label{chap:codechanges}

The changes that had the greatest impact on the behavior were the collector integration and the allocation re-scaling. The collector code is available on the link at the references, therefore, only the usage of the package will be inserted here in comparison to the original.

\lstset{language=Java,
             basicstyle=\footnotesize,
             numbers=left,
             extendedchars=\true
             showspaces = false,
             numberstyle=\footnotesize,
             frame=shadowbox,
             breaklines = true}

Starting with the original NodeManager creation, in which the totalResources are gotten. Note how the memoryMB and virtualCores variables are taken from the conf, which is the pointer to the default xml file. This method is from the NodeStatusUpdaterImpl class.

\begin{lstlisting}

protected void serviceInit(Configuration conf) throws Exception {
    int memoryMb = 
        conf.getInt(
            YarnConfiguration.NM_PMEM_MB, YarnConfiguration.DEFAULT_NM_PMEM_MB);
    float vMemToPMem =             
        conf.getFloat(
            YarnConfiguration.NM_VMEM_PMEM_RATIO, 
            YarnConfiguration.DEFAULT_NM_VMEM_PMEM_RATIO); 
    int virtualMemoryMb = (int)Math.ceil(memoryMb * vMemToPMem);
    
    int virtualCores =
        conf.getInt(
            YarnConfiguration.NM_VCORES, YarnConfiguration.DEFAULT_NM_VCORES);

    this.totalResource = recordFactory.newRecordInstance(Resource.class);

    this.totalResource.setMemory(memoryMb);
    this.totalResource.setVirtualCores(virtualCores);
	metrics.addResource(totalResource);
    this.tokenKeepAliveEnabled = isTokenKeepAliveEnabled(conf);
    this.tokenRemovalDelayMs =
        conf.getInt(YarnConfiguration.RM_NM_EXPIRY_INTERVAL_MS,
            YarnConfiguration.DEFAULT_RM_NM_EXPIRY_INTERVAL_MS);
    
    // Default duration to track stopped containers on nodemanager is 10Min.
    // This should not be assigned very large value as it will remember all the
    // containers stopped during that time.
    durationToTrackStoppedContainers =
        conf.getLong(YARN_NODEMANAGER_DURATION_TO_TRACK_STOPPED_CONTAINERS,
          600000);
    if (durationToTrackStoppedContainers < 0) {
      String message = "Invalid configuration for "
        + YARN_NODEMANAGER_DURATION_TO_TRACK_STOPPED_CONTAINERS + " default "
          + "value is 10Min(600000).";
      LOG.error(message);
      throw new YarnException(message);
    }
    if (LOG.isDebugEnabled()) {
      LOG.debug(YARN_NODEMANAGER_DURATION_TO_TRACK_STOPPED_CONTAINERS + " :"
        + durationToTrackStoppedContainers);
    }
    super.serviceInit(conf);
    LOG.info("Initialized nodemanager for " + nodeId + ":" +
        " physical-memory=" + memoryMb + " virtual-memory=" + virtualMemoryMb +
        " virtual-cores=" + virtualCores);
  }
\end{lstlisting}

Then the changes made in the method to enable collectors. The rest of the method was not altered. The reason for the double casting is that the collector returns a Float and Double value and it's not possible to cast directly to int.

\begin{lstlisting}
protected void serviceInit(Configuration conf) throws Exception {
	PhysicalMemoryCollector memoryCollector = new PhysicalMemoryCollector();
    TotalProcessorsCollector processorsCollector = new TotalProcessorsCollector();

    int memoryMb = (int)(float)memoryCollector.collect().get(0)/1024;
    int virtualCores = (int)(double)processorsCollector.collect().get(0);

    this.totalResource = recordFactory.newRecordInstance(Resource.class);
\end{lstlisting}

The other change that had a strong impact in CapacityScheduler behavior was the insertion of recalculations of allocation limits inside the addNode method. This method belongs to the CapacityScheduler class. Starting with the original code.

\begin{lstlisting}
private synchronized void addNode(RMNode nodeManager) {
    this.nodes.put(nodeManager.getNodeID(), new FiCaSchedulerNode(nodeManager,
        usePortForNodeName));
    Resources.addTo(clusterResource, nodeManager.getTotalCapability());
    root.updateClusterResource(clusterResource);
    ++numNodeManagers;
  }
\end{lstlisting}

The same changes made on the addNode were also made on removeNode. Thus, when a node is killed, or is not accessible for a long period, it will be removed and the limits will be adjusted too.

\begin{lstlisting}
private synchronized void addNode(RMNode nodeManager) {
    this.nodes.put(nodeManager.getNodeID(), new FiCaSchedulerNode(nodeManager,
        usePortForNodeName));
    Resource oldCap = Resources.clone(clusterResource);
    Resources.addTo(clusterResource, nodeManager.getTotalCapability());
    root.updateClusterResource(clusterResource);
    ++numNodeManagers;
    LOG.info("MEU Added node " + nodeManager.getNodeAddress() +
        " clusterResource before: " + oldCap + " nodecapability: " + nodeManager.getTotalCapability() + " clusterResource now: " + clusterResource);
    LOG.info("MEU Changing allocation minimum & maximum. Actual minimum: " + this.minimumAllocation + "actual maximum: " + this.maximumAllocation + ".\nDefault settings: cluster must have capacity for at least " + minimumContainers + " containers, and no more than " + maximumContainers + "containers. 8GB RAM cluster would have 1GB minimum/maximum, 80GB RAM cluster would have 4GB minimum and 10GB maximum.");
    this.minimumAllocation.setMemory(clusterResource.getMemory() / maximumContainers);
    this.minimumAllocation.setVirtualCores(clusterResource.getVirtualCores() / maximumContainers);
    this.maximumAllocation.setMemory(clusterResource.getMemory() / minimumContainers);
    this.maximumAllocation.setVirtualCores(clusterResource.getVirtualCores() / minimumContainers);
    if (this.minimumAllocation.getMemory() < minimumMemory)
       this.minimumAllocation.setMemory(minimumMemory);
    if (this.minimumAllocation.getVirtualCores() < minimumVcores)
       this.minimumAllocation.setVirtualCores(minimumVcores);
    if (this.maximumAllocation.getMemory() < this.minimumAllocation.getMemory())
       this.maximumAllocation = this.minimumAllocation;
    LOG.info("MEU New minimumAllocation settings: " + minimumAllocation + "\nNew maximumAllocation settings: " + maximumAllocation);
\end{lstlisting}

%%=============================================================================
%% Anexos
%%=============================================================================
%\annex
%
\chapter{Questionários aplicados para a Análise de Tarefas}



%
\chapter{Resultados da Análise de Casos de Uso}
\begin{figure}[hbtn]
   \centering
   \includegraphics[width=6.5cm]{figuras/figura09-anexob.eps}
   \caption{Análise dos pincipais Portais da área de Informática de instituições federias}
   \label{fig:CASOSDEUSO}
\end{figure}

\end{document}
