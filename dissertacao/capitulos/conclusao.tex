\chapter{Conclusion and future work}
%-------------------------------------------------------------------
This work had the objective of improving Hadoop scheduling, during the study process it was identified that the CapacityScheduler had already the base for a context-aware scheduling. However, this schedules lacked some fundamental components in order to be context-aware, such as NodeManager real resource information and allocation limits scaling with the total cluster capacity.

Through development of this work changes have been made on original source code, these changes allowed Hadoop to be more aware of the context of nodes composing the cluster. The scheduling algorithm remained the same, however key limitations caused by Hadoop's default configurations were noticed. A new distribution containing a context-aware CapacityScheduler was generated in order to solve these issues.

The context-aware CapacityScheduler is capable of receiving the real capacity from each NodeManager, thanks to the collector plugged on NodeManager. This provides the cluster a better scaling potential while also using every node's full capacity. Using the context-aware CapacityScheduler, the allocations can be made to the full potential of the cluster instead of waiting for more resources when the cluster actually had almost 40GB of free memory per node.

Although the context-aware CapacityScheduler has better scaling potential and solves some problems on containers management, all contributions made are purely static and there are more ways to impact and improve Hadoop scheduling. Given Hadoop high modularity, it is possible to improve scheduling changing many areas that range from ApplicationMaster and Queues to NodeManager HeartBeat behavior.

Following there are some suggestions of future work:

\begin{itemize}
\item Extending the Resource class so it can track more resources like CPU load.
\item Improving CapacityScheduler scheduling, taking into account other resources information.
\item Modification of ApplicationMaster behavior.
\item Implementation of a scheduler capable of starting containers directly on a NodeManager, and not dependant on queues.
\end{itemize}
