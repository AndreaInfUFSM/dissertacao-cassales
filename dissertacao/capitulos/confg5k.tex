\chapter{Grid'5000 execution environment configuration}

This appendix has the necessary steps in order to setup a correctly working execution environment on Grid'5000 cluster.

The Apache Hadoop's installation, contrary to end users standard program installations, doesn't have a graphical interface. Actually the installation is just the extraction of files to a determined folder, however the environment configuration isn't trivial and demands some network administration knowledge.

For the correct functioning of the framework, it is necessary to edit of some files responsible for describing the environment in which it will execute. Besides, it is necessary to have a network in which every node has access to the others. Knowing that Apache Hadoop has suffered some heavy changes in changing from 1.x to 2.x, it was expected that installing both versions would make the differences clearer.

In the beginning of this work the objective was to install and configure the Apache Hadoop in three situations:
\begin{itemize}
        \item localhost, a single-node case.
        \item mini cluster, a multi-node case.
        \item Grid'5000, another multi-node case.
\end{itemize}

There are two configuration steps, the network step which refers to the configuration that doesn't depend on Hadoop and the Hadoop step which refers to the parameters that influence Hadoop's execution. This step is composed of tasks executed on the operational system, such as the user creation and ssh configuration. The second step, on the other hand, is the proper Hadoop environment configuration and, therefore, it has to change some configuration files in a way that Hadoop can be executed without errors.

The OS configuration is equal on both versions, since it has already been noted that it is independent of Hadoop. The Hadoop configuration, however, has a few differences between versions. Even so, both versions follow the same general rules. The Hadoop configuration is made through a bunch of xml files already mentioned in section \ref{chap:fundamentacao}, subsection \ref{sec:envconfig}, in which some properties containing name and value are inserted. This properties will be responsible for changing the default Hadoop behavior, but aside some properties changing their names, the biggest difference is that YARN version has a new xml file named \textit{yarn-site.xml}, it contains all parameters related to the YARN framework and therefore has a high influence on MapReduce and job execution performance.

Initially it was chosen to configure both versions on every specified instance, however, given the project focus the version 1.x was no longer a viable alternative and it's use was discontinued.

The localhost environment configuration, is a simple process and occurred  without problems after a basic contact with the framework. The mini cluster environment, was configured using two machines, already presenting a difference, in which the master could also be a slave on 1.x, in other words, it was possible to run NameNode and DataNode on the same node. On YARN version, the NameNode machine can't start a DataNode service, and the same is valid for ResourceManager and NodeManager. This change makes the tasks of processing and management totally apart and differentiable making the cluster more organized.

The real experiment starts when the Hadoop is deployed on a real cluster, like the Grid'5000, it was decided to use only the YARN version from this point onward. The installation of YARN requires only a few changes from mini cluster to Grid'5000 environment. The Grid also supplies a lot of tools that makes the job easier for Hadoop management.

At the end of this stage, some Hadoop peculiarities have already been cleared and the execution environment for posterior testing of the implementations was already deployed. Also, it is important to note that it was possible to identify some contexts that would present a high difficult to change the behavior. One of these behaviors would be the addition or removal of nodes on real time after the cluster initialization. The difficult comes from the way the service uses static reference to xml files that are read only on the initialization, making it impossible to use their values without restarting the services.